\section{Kommissorium for BXX-eu}
(Bestyrelsen år XX evaluerings udvalg)\\

\textbf{Sammensætning}
Udvalget består af to til tre personer, som godkendes af Fællesrådet.\\
Udvalgets medlemmer skal være medlemmer af PF.\\
Ingen af udvalgets medlemmer må være en del af den fungerende Bestyrelse eller kandidater til BXX.\\
Det er ikke et krav at udvalgets medlemmer har siddet i en tidligere bestyrelse for Polyteknisk Forening, men gruppen skal samlet have et kendskab til det at sidde i PF’s bestyrelse. Det bør tilstræbes at udvalgets medlemmer kender til bestyrelsesarbejde, særligt inden for PF, og har erfaring i gruppedannelse og samarbejde.\\
Det er et krav at udvalgets medlemmer kender Polyteknisk Forenings områder godt. Områderne afspejler de politiske, sociale og økonomiske aspekter af foreningen, samt arbejdsformer og struktur.\\
\\
\textbf{Funktionsperiode}
Udvalget vælges senest på det sidste FR-møde i forårssemesteret og varetager sin funktion derfra og til det konstituerende FR-møde. Jf. \textbf{§ S 13 ???} i PFs Statutter.\\
\\
\textbf{Formål}
Udvalgets formål er at hjælpe kandidater til bestyrelsen optil valget og derefter støtte den nyvalgte bestyrelse indtil
deres tiltræden.
\\
\\
\textbf{Arbejdsopgaver}
\begin{list}{•}
\item Udvalget skal stå for opkvalificeringen af BXX evt. i samarbejde med den siddende bestyrelse.
\item Sikre at BXX lægger en fornuftig plan for sit arbejde frem til konstitueringen. Herunder at BXX tidligt i forløbet sættes ind i Polyteknisk Forenings områder, samt får tid til at diskutere politik, struktur og visioner som BXX måtte have i forhold til foreningen.
\item Være opmærksom på hvordan BXXs samarbejdsform og sammenhold fungerer. Dette bør gøres ved personlige
        samtaler, samt ved at holde jævnlig kontakt med BXX.
\item Ved alvorlige konflikter, efter udvalgets vurdering, er det udvalgets opgave at hjælpe BXX med synliggørelsen
     af konfliktens art, samt være tredjepart i problemløsningen.
\item Udvalget skal være rådgivende og kritisk i forbindelse med BXXs udarbejdelse af visionsoplægget. Kritikken
      må kun foregå med baggrund i foreningens politik og struktur, og må således ikke være kendetegnet for
          enkeltmedlemmers, i udvalget, personlige overbevisning.
\item Skal opfordre til at der sker en overlevering fra den fungerende bestyrelse til BXX.
\item I forbindelse med konstitueringen af Bestyrelsen skal udvalget redegøre for BXX’s arbejde og give en
      evaluering af BXX-forløbet.
\end{list}
\textbf{Kompetence}\\
I det følgende er udvalgets beføjelser og referencerammer beskrevet.
\begin{list}{•}
\item Udvalget skal til enhver tid referere tilbage til Fællesrådet, enten på eget initiativ eller på opfordring fra Fællesrådet
         selv.
\item Udvalget har ret til at rådgive BXX, både enkeltpersoner og hele gruppen. Dette gælder både i forhold til resten af
       foreningen og i forhold til BXX internt.
\item Udvalget har beføjelse til at henstille Bestyrelsen om ikke at blande sig i BXXs arbejde og politikformulering, hvis
    der er mistanke om indoktrinering fra Bestyrelsens side. I tilfælde af at Bestyrelsen ikke følger henstillingen, kan
     udvalget tage sagen op i Fællesrådet.
\item Udvalgets medlemmer må ikke bevidst præge BXX i forhold til personlige holdninger omkring PFs områder.
\item      Udvalget har til enhver tid tavshedspligt, i forhold til enkeltpersoner i BXX, når det drejer sig om personfølsomme
     oplysninger. Dette gælder også over for Fællesrådet.  
\end{list}


%På baggrund af dette kommissorium vælges BXX-eu ved et Fællesrådsmøde som ligger før nedsættelsen af BXX.\\

Oplysninger af fortrolig eller følsom karakter, som medlemmer af BXX-eu er kommet i besiddelse af i forbindelse med arbejdet, må ikke viderebringes til tredjepart efter BXX-eu har afsluttet sit arbejde. 
\\
\\
\textit{Godkendt efter revidering på Fællesrådsmøde 155.}
