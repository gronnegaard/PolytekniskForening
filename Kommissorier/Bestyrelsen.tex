\section{Kommissorium for bestyrelsesmedlemmer}
\textbf{Formanden}
\begin{enumerate}
\item  Skal være foreningens officielle ansigt internt og eksternt.
\item Skal medvirke i udviklingen af foreningen.
\item Skal sikre, at bestyrelsen jævnligt gør status over sit arbejde.
\item Skal have det endelige ansvar for et godt samarbejde i bestyrelsen.
\item Skal have overblik over og holde sig orienteret om:
\begin{itemize}\addtolength{\itemsep}{-0.5\baselineskip}
\item Foreningens politikker og holdninger.
\item Aktiviteter i bestyrelsen og i foreningen.
\item Foreningens økonomi.
\item Aktiviteter og aktuelle sager på DTU.
\item Landspolitik
\end{itemize}
\item Skal deltage i alle politikskabende møder
\end{enumerate}
\textcolor{white}{linjeskift :)}\\
\textbf{Forretningsrådsformanden}
\begin{enumerate}
\item Skal have et indgående kendskab til foreningens økonomi.
\item Skal sørge for at budget, regnskab, kritisk revision og årsrapport bliver udfærdiget.
\item Skal have ugentlig kontakt med foreningens ansatte.
\item Skal sikre afholdelse af minimum 4 møder med Forretningsrådet, samt at dette er velfungerende.
\end{enumerate}
\textcolor{white}{linjeskift :)}\\
\textbf{Bestyrelsen}
\begin{enumerate}
\item Skal være gode ambassadører for PF samt bidrage til forbedring og udvikling af foreningen.
\item Skal uddelegere relevante arbejdsopgaver og aktiviteter enten internt i bestyrelsen eller i foreningen.
\item Skal afholde ordinært bestyrelsesmøde en gang om ugen i undervisningsperioden og have et velfungerende internt samarbejde.
\item Skal dokumentere deres arbejde og beslutninger.
\item Skal søge, at der findes velfungerende faglige råd på alle retninger, og så vidt muligt deltage til alle faglige rådsmøder.
\item Skal være repræsenteret dagligt i studentercenteret og aflaste PFS i tidsrummet kl. 10 – 14.
\item Den konstituerede bestyrelse, som et hele, skal have erfaringer indenfor hele PF.
\end{enumerate}
