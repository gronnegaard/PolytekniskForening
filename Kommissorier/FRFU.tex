\section{Kommissorium for Fællesrådets forretningsudvalg}
Fællesrådets forretningsudvalg (FRFU) har til opgave at sikre at PF’s love og statutter bliver overholdt i forbindelse
med afholdelse af og indkaldelse til Fællesrådsmøder, samt at fællesrådets (FR) arbejde gennem året bliver planlagt og
udført på en fornuftig måde. FRFU består af 3-6 personer hvoraf mindst én er medlem af FR og mindst én er medlem af
Bestyrelsen. Medlemmer af FRFU, som ikke er medlemmer af Fællesrådet, har lov at deltage i behandlingen af lukkede
punkter.\\
\\
\textbf{Opgaver og kompetencer}
\begin{list}{•}
\item FRFU har ansvaret for, at der senest 10 dage før ethvert ordinært FR-møde udsendes en foreløbig dagsorden, samt at der udsendes en endelig dagsorden senest 3 dage før FR-mødet.
\item FRFU skal forud for alle ordinære FR-møder sørge for at der bliver afholdt et møde hvor den endelige
dagsorden diskuteres evt. sammen med personer, der har leveret oplæg til FR-mødet og dirigenterne.
\item Det er FRFU’s ansvar, at FR’s mailingliste er opdateret. På denne liste er ordinære FR-medlemmer,
observatører samt PF’s-bestyrelse. FRFU har bemyndigelse til at sætte andre personer på listen, hvis de
skønner det er relevant. Hvis personerne ikke er medlem af PF skal FR orienteres.
\item Det er FRFU’s ansvar at opdatere principkataloget.
\item Det er FRFU’s ansvar, at PF’s arkiver om FR’s aktiviteter er opdaterede og fuldstændige.
\item FRFU har ret til at indkalde FR til ekstraordinære møder efter gældende regler i PF’s love.
\item FRFU indstiller referent(er) og dirigent(er) til FR ved hvert FR-møde.
\end{list}
\vspace{1cm}
Alt fortroligt materiale udsendes ikke på mailinglisten
