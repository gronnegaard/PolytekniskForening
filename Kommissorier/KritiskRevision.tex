\section{Kommissorium for Kritisk Revision}

\begin{list}{•}
\item Der skal foretages én kritisk revision hvert år.
\item Den kritiske revision skal tage stilling til om pengene er brugt i overensstemmelse med foreningens formålsparagraf.
\item Det er Fællesrådet som ved en fast procedure skal nedsætte de kritiske revisorer
\end{list}

Sammensætning:
\begin{list}{•}
\item Kritisk revision består af 2-3 personer.
\item Flertallet af de kritiske revisorer må ikke have udarbejdet, eller medvirket ved udarbejdelsen af budget eller regnskab, dvs. den siddende og den foregående bestyrelse, samt det foregående og det siddende fællesråd. Personerne skal være nært knyttet til PF eller have et grundigt kendskab til PF og vores forretningsgang.
\item Alle kritiske revisorer må ikke have siddet i den nuværende eller foregående bestyrelse.
\end{list}
Opgaver og pligter:
\begin{list}{•}
\item Den kritiske revision skal behandle økonomien i PF. De skal gennemgå regnskaberne og sætte dem overfor
budgetterne, for at se om pengene er brugt som besluttet. Det er altså primært budgetafvigelser, der er
interessante og skal kommenteres.
\item De kritiske revisorer skal undersøge om Fællesrådets beslutninger, mht. PFS økonomi, er overholdt og udført.
Desuden skal de checke, om bestyrelsen har udført deres opgaver i den økonomiske sektor, og derved holdt
hvad de har lovet.
\item De kritiske revisorer skal undersøge om der er foretaget fejlarkiveringer, eller foretaget svindel, evt. ved
stikprøvekontrol.
\item De kritiske revisorer skal kontakte PFs bestyrelse for at få udleveret materiale til deres arbejde. Bestyrelsen
skal fremskaffe det eller give adgang til dette. De kritiske revisorer bør finde svar på uklarheder ved at rette
henvendelse til Fællesrådsmedlemmer, Polyteknisk Regnskabsafdeling (ved forretningsrådsformanden) eller
bestyrelsesmedlemmer
\item De kritiske revisorer skal aflevere resultatet af deres arbejde skriftligt, så det kan behandles på
Fællesrådsmødet før det afsluttende Fællesrådsmøde.
\item Den skriftlige rapport opdeles i en lukket og en åben del, således at resten af foreningen, herunder de faglige
råd, kan få kendskab til evt. problemer påpeget af kritisk revision, dog under hensyntagen til behovet for at
hemmeligholde visse dele af rapporten.
\item De kritiske revisorer har tavshedspligt.
\end{list}

\textit{Vedtaget af FR på FR165 d. 10. april 2008.}
