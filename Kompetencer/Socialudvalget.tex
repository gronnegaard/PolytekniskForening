\section{Socialudvalgets formål og kompetencer}
\textbf{Formål}\\
Socialudvalgets ene formål er at diskutere og formulere PF’s politik på levevilkårsområdet og søge at forbedre de
fysiske og sociale rammer af studiet, herunder udvikling og
vedligeholdelse.
Socialudvalgets andet formål er at arrangere og afholde sociale events på DTU.\\
\\
\textbf{Kompetencer}\\
\underline{Levevilkårspolitik}\\
Socialudvalget har kompetence til at fastlægge foreningens levevilkårspolitik.
Fællesrådet kan naturligvis vælge at tage en af udvalgets beslutninger op og omgøre
denne, men det er Fællesrådets ansvar at tage sagen op.
Konkrete emneområder, som udvalget behandler og formulerer PF-politikker for, er alle
områder, der har med levevilkårene for studerende (ved DTU) at gøre:
\begin{itemize}
\item Boligforhold
\item SU og de studerendes økonomiske forhold
\item Studie-/undervisningsmiljø
	\begin{itemize}
	\item Studentersociale forhold/tilbud
	\item Undervisningslokaler og fysiske faciliteter for studerende på DTU
	\item Campus
	\item Transport til og fra uddannelsesstedet
	\end{itemize}
\end{itemize}

Socialudvalget er baggrundsgruppe for Socialudvalgsformanden samt andre sociale
grupper, under PF.\\
\\
\underline{Events}\\
Socialudvalget har kompetence til at planlægge og afholde sociale events.
Socialudvalget er ansvarlig for afholdelse af følgende events på faste tidspunkter på året:
\begin{itemize}
\item PF fodboldturnering, afholdes en lørdag i 3-ugers perioden i juni.
\item PF motionsløb, sidste torsdag og fredag inden DTU's efterårsferie.
\item PF skitur, i DTU's vinterferie.
\end{itemize}
\textcolor{white}{linjeskift :)}\\
\textbf{Medlemmer}\\
Alle interesserede kan blive medlem ved udvalgets konstituerende møde. Øvrige kan godkendes af Socialudvalget. Det
tilstræbes at have repræsentanter fra kollegierne, alle de faglige råd samt S-Huset.

Ordinære medlemmer (med stemmeret) er to fra hvert af de faglige råd og to fra DFK (De Forenede Kollegier).

Fællesrådet godkender Socialudvalgets endelige sammensætning.\\
\\
\textbf{Møder}\\
Alle møder er åbne, og enhver, der møder op til et møde i udvalget, har taleret. Alle
medlemmer af udvalget kan bede udvalget om at tage en sag op.

Socialudvalgsformanden er ansvarlig for at udarbejde dagsordenen, og det er ligeledes denne, der har ansvaret for at
indkalde til møderne.

Alle møder skal indkaldes med mindst 14 dages varsel og ligge i DTU's undervisningsperioder.

Senest seks dage før et møde skal der foreligge en endelig dagsorden, og alle bilag skal desuden være offentliggjort, såfremt muligt.

Et møde er beslutningsdygtigt i henhold til de almindelige regler i PF’s love og statutter.\\
\\
\textbf{Udvalg indstillet til af Socialudvalget}\\
Socialudvalget har på det konstituerende møde ret til at indstille medlemmer til udvalg med relation til foreningen på
levevilkårsområdet. Socialudvalget er repræsenteret i disse udvalg med et fast antal pladser, bestemt af udvalgene selv.
Disse udvalg er følgende:
\begin{itemize}
\item DTU’s Kantineudvalg (3 pladser)
\item KKO’s bestyrelse (2 pladser)
\item PF’s Indstillingsudvalg, PFIU (9 pladser og Socialudvalgsformanden)
\item DTU’s Koncern Arbejdsmiljøudvalget, KAMU (2 pladser, et medlem fra forrige og  et fra den siddende bestyrelse)
\item Danske Studerendes Fællesråd’s Levevilkårsudvalg, DSF-LU, (2 pladser)
\item Studiemiljøudvalget, SMU (Socialudvalgsformanden og den forgående Socialudvalgsformand)
\end{itemize}

Repræsentanterne i disse udvalg forventes at holde Socialudvalget opdateret løbende med nyt fra udvalget. Dette indgår
som faste punkter på Socialudvalgets mødedagsorden.