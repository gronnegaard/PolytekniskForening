\section{Uddannelsespolitisk Råds kompetencer og forretningsorden}
Udvalget er sammensat af fællesrådet (FR) gennem FRs opstilling af repræsentanter til studienævn, uddannelsesudvalg, akademisk råd samt DTU’s bestyrelse.\\
\\
\textbf{Udvalgets formål}\\
Udvalgets formål er at diskutere og fastlægge PF's politik på det uddannelsespolitiske område samt at koordinere PF's
indsats i DTUs styrende organer.\\
\\
\textbf{Udvalgets kompetencer}\\
Udvalget har kompetence til at fastlægge PFs uddannelsespolitik indenfor DTU, FR kan naturligvis vælge at tage en af
udvalgets beslutninger op og omgøre denne, men det er FR's ansvar at tage sagen op.
\\
\\
De konkrete emneområder som udvalget kan formulere politikker for, er alle områder der har direkte med DTU's
uddannelser og undervisning at gøre. Herunder flagmodellen for civilbachelorer og CDIO konceptet for diplombachelorer, DTU’s uddannelsesudbud, PF’s holdning til hvad ingeniørkompetencer er osv.
\\
\\
\textbf{Medlemmer}
\\
Alle medlemmer og suppleanter i studienævn, uddannelsesudvalg, akademisk råd samt DTU’s bestyrelse valgt på PF-lister er medlemmer af udvalget og har dermed stemmeret til udvalgets møder. Et institut kan højst bære det antal pladser der er i studienævnet. Således har suppleanter kun stemmeret i det omfang et institut ikke møder fuldtalligt op.
\\
\\
\textbf{Møder}\\
Alle møder er åbne og enhver der møder op til et møde i udvalget har taleret. Alle medlemmer af udvalget kan bede
udvalget om at tage en sag op.

Det er PF's uddannelsespolitiske koordinator der har ansvaret for at sætte punkter på dagsordnen og det er ligeledes
denne der har ansvaret for at indkalde til møderne.

Alle møder skal indkaldes med mindst 14 dages varsel og ligge i DTU's 13 ugers periode eller i DTUs 3-ugerperiode.
Senest en uge før et møde skal der foreligge en endelig dagsorden og alle bilag skal desuden være offentliggjort.

Et møde der er korrekt indkaldt og hvor mindst halvdelen af alle institutter er repræsenteret er beslutningsdygtigt.
