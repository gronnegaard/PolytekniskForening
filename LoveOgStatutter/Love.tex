\setcounter{section}{12}
\section*{Love for Polyteknisk Forening}
\addcontentsline{toc}{section}{Love for Polyteknisk Forening}

\begin{list}
{\S L.\arabic{fpara}}{\usecounter{fpara}}

%%%%%%%%%%%%%%%%%%%%%%%%%%%%%%%%%%%



\subsection{Formål}\label{L:kap:Formaal}
\item Foreningens formål er at repræsentere de studerende ved Den Polytekniske Læreanstalt, Danmarks Tekniske Universitet (DTU) over for denne og udadtil. Dette omfatter deres faglige, sociale og klubmæssige interesser.

\stk{2} Foreningens hele virke er uafhængigt af partipolitiske og religiøse interesser.

\subsection{Medlemmerne}
\label{L:kap:medlemmerne}
\item Som ordinært medlem kan optages enhver af de ved DTU indskrevne studerende samt adgangskursister.
\item De ved DTU indskrevne studerende har ret til stemmeafgivning og kandidatur ved valg til foreningens faglige råd.
\item De ved DTU indskrevne ph.d.-studerende kan optages som ph.d. medlemmer.
\item Studerende ved Maskinmester Skolen København (MSK) som er medlem af Studieforeningen ved MSK kan opnå et associeret medlemskab af PF.
\item Undervisere ved og ingeniører udgået fra de ingeniørvidenskabelige læreanstalter samt medlemmer af         Ingeniørforeningen i Danmark (IDA) kan blive seniormedlemmer. Øvrige enkeltpersoner kan efter særskilt godkendelse af Fællesrådet blive seniormedlemmer.
\item \label{L:Medlem:Pligt} Alle medlemmer har pligt til at indordne sig under foreningens love og statutter samt pligt til at betale et af Fællesrådet fastsat kontingent.

\stk{2}
Fællesrådet, jvf. \kapref{L:kap:faellesraadet}, kan på opfordring fra Bestyrelsen, jvf. \kapref{L:kap:bestyrelsen} eller på eget initiativ, ekskludere medlemmer der ikke indordner sig under foreningens love og statutter, udøver vold og hærværk eller på anden måde skader foreningen og dens omdømme.

\stk{3}
Eksklusion af medlemmer kan kun ske såfremt dette vedtages på to, på hinanden følgende, fællesrådsmøder, hvoraf et kan være ekstraordinært, med mindst 3 ugers mellemrum. Eksklusion kan kun ske såfremt mindst 2/3 af de tilstedeværende Fællesrådsrepræsentanter enes herom, og mindst halvdelen af
samtlige stemmeberettigede er tilstede.

\stk{4}
Ekskluderede medlemmer kan optages i foreningen igen såfremt dette vedtages på to, på hinanden følgende, ordinære fællesrådsmøder, hvoraf et kan være ekstraordinært, med mindst 3 ugers mellemrum. Genoptag i foreningen kræver 2/3 flertal blandt de tilstedeværende Fællesrådsrepræsentanter, og at mindst halvdelen af samtlige stemmeberettigede er tilstede. Genoptag i foreningen kan dog tidligst ske et halvt år efter eksklusion.

\stk{5}
Medlemmer der er indstillet til eksklusion har ret til at tale og forsvare sig til de fællesrådsmøder hvor sagen bliver behandlet, men er i alle andre henseender suspenderet fra Polyteknisk Forening indtil sagen er færdigbehandlet.
\item Medlemmer af foreningen hæfter ikke for foreningens økonomi.

\item \label{L:medlem:PHD} Ph.d.-studerende og medlemmer med associeret medlemskab repræsenteres ikke politisk af Polyteknisk Forening.

\stk{2}
Ph.d.-studerende og medlemmer med associeret medlemskab kan ikke blive studiemedlemmer af Ingeniørforeningen Danmark gennem PF.

\item Et associeret medlemskab indebærer benyttelse af PF’s medlemsrabatter, rabat på fester og koncerter, sociale events, samt mulighed for være del af, allerede eksisterende, PF klubber.

\subsection{Fællesrådet}	
\label{L:kap:faellesraadet}
\item Fællesrådet er foreningens højeste myndighed og skal vedtage hovedlinjerne i foreningens dispositioner. Fællesrådet er det koordinerende organ for de faglige råds arbejde.

\item \label{L:FR:ValgtilFR} Hvert fagligt råd vælger 2 ordinære medlemmer samt 1 suppleant til Fællesrådet. Dog kan faglige råd, der har mindst 17 ordinære medlemmer, og på hvis område der de to foregående år har været et gennemsnitligt årligt optag på mindst 100 studerende, vælge 3 ordinære medlemmer og 1 suppleant til Fællesrådet.\\

\stk{2} Alle ordinære medlemmer samt suppleanter til Fællesrådet ska være medlemmer af Polyteknisk Forening.

\stk{3} Fællesrådets ordinære medlemmer og suppleanter vælges i øvrigt af de faglige rådi overensstemmelse med  \kapref{kap:valgtilfaellesraad}.

\stk{4} Ny oprettede faglige råd har stemmeret i Fællesrådet i overensstemmelse med \kapref{kap:valgtilfaellesraad}.

\item Fællesrådet vælger 2-3 kritiske revisorer. De skal være medlemmer af Polyteknisk Forening og de må ikke sidde i den nuværende eller den foregående bestyrelse eller forretningsråd. 

\item Fællesrådet vælger eller bemyndiger bestyrelsen eller forretningsrådet til at vælge repræsentanter til de foreninger, bestyrelser og lignende, som Polyteknisk Forening har indstillingsret til. Alle disse repræsentanter skal være nuværende eller tidligere medlemmer af Polyteknisk Forening, dog kan medlemmer, der er ekskluderet i henhold til \lref{L:Medlem:Pligt} ikke vælges.

\stk{2} Alle ordinære medlemmer samt suppleanter til Fællesrådet skal være medlemmer af Polyteknisk Forening.\\

\stk{3} Fællesrådets ordinære medlemmer og suppleanter vælges i øvrigt af de faglige råd i overensstemmelse med \kapref{kap:valgtilfaellesraad}.

\stk{4} Ny oprettede faglige råd har stemmeret i Fællesrådet i overensstemmelse med \kapref{kap:valgtilfaellesraad}.

\item Fællesrådet vælger 2-3 kritiske revisorer. De skal være medlemmer af Polyteknisk Forening og de må ikke sidde i den nuværende eller den foregående bestyrelse. 

\item Fællesrådet vælger eller bemyndiger bestyrelsen til at vælge repræsentanter til de foreninger, bestyrelser og lign., som Polyteknisk Forening har indstillingsret til. Alle disse repræsentanter skal være nuværende eller tidligere medlemmer af Polyteknisk Forening, dog kan medlemmer der er ekskluderet i henhold til \lref{L:Medlem:Pligt} ikke vælges.

\item Fællesrådet opretter og nedlægger udvalg efter behov. Dog kan lovfæstede udvalg kun nedlægges ved en ændring af lovene. Alle ordinære medlemmer af Fællesrådet har ret til at deltage i alle udvalgsmøder, herunder i lovfæstede udvalg på lige vilkår med ordinære medlemmer dog uden stemmeret.

\item Ethvert Fællesrådsmedlem skal have lejlighed til ved selvsyn at sætte sig ind i foreningens arkiver.

\item Ethvert medlem af Polyteknisk Forening har normalt ret til at overvære Fællesrådsmøder jvf. \sref{S:MoederGenerelt:fortrolig} og har taleret til disse.

\item Ethvert medlem af Fællesrådet skal have lejlighed til at sætte sig ind i foreningens regnskab.



%\subsection*{Kap. L.4 De faglige råd}
\subsection{De faglige råd}
\label{L:kap:Faglige}

\item \label{L4:hvem} Der kan i PF oprettes et fagligt råd korresponderende med
	\begin{enumerate}
	\item Optagelsesområde på DTUs bacheloringeniørstudie.
	\item Optagelsesområde på DTUs diplomingeniørstudie.
	\item Levnedsmiddelsuddannelsen på DTU.
	\item Optagelses område på en DTU kandidat, hvor der ikke er en korresponderende civil eller diplom bacheloruddannelse. Ved tvivl om eksistensen af en korresponderende bachelor uddannelse konsulteres det siddende fællesråd.
	\end{enumerate}     
    
\item \label{L4:ordinaere} De faglige råd har som ordinære medlemmer et antal på mindst 7 personer.

\stk{2} Ingen kan være ordinært medlem af to eller flere faglige råd % \label{L4:ordinaere1} ændre i ref stedet til \lref{L4:ordinaere} stk. 2.

\stk{3} Kun studerende, hvis primære faglige og studiemæssige tilknytning repræsenteres bedst af det faglige råd for et optagelsesområde, kan sidde som medlemmer af det faglige råd.
	
\stk{4} Alle studerende kan dog være medlem af det faglige råd, tilknyttet deres bacheloruddannelse.

\item \label{L4:flereraad} Såfremt der ikke kan skaffes nok medlemmer til at oprette et fagligt råd for et optagelses område, er det tilladt at oprette et fagligt råd der dækker to eller flere optagelsesområder. Dette faglige råd vil i alle andre henseender fungere som et råd for ét optagelsesområde. Herunder kun have repræsentanter svarende til et råd i PF’s organer.

\item \label{L4:koordinering} De faglige råd koordinerer, repræsenterer og arbejder for de studerendes interesser på de specifikke faglige områder.

\item Enhver studerende ved DTU har ret til at overvære møderne i de faglige råd og har taleret til disse møder, i overensstemmelse med \sref{S:MoederGenerelt:stemmeret}.

\item På det konstituerende møde i hvert af de faglige råd vælger de ordinære medlemmer en økonomiansvarlig og én rådsformand, som varetager kontakten til foreningens bestyrelse.

\item \label{L:faglige:DTUvalg} I forbindelse med valg til de styrende organer ved DTU sørger Fællesrådet for den endelige opstilling af Polyteknisk Forenings kandidater efter indstilling fra de faglige råd.

\item Eventuelle tvistigheder mellem to eller flere faglige råd angående prioriteringen af valglister jf. \lref{L:faglige:DTUvalg} ved valget til de styrende organer bringes for Fællesrådet. Fællesrådet kan vælge at kræve sagen genbehandlet i de specifikke faglige råd som et åbent punkt.


%%%%%%%%%%%%%%%%%%%%%%%%%%%%

\subsection{Bestyrelsen}
\label{L:kap:bestyrelsen}
\item \label{L:best:valg} Fællesrådet vælger ved personvalg, på et ordinært fællesrådsmøde i november eller december, de kommende bestyrelsesmedlemmer, der skal tiltræde på det konstituerende fællesrådsmøde i februar. 

\stk{2} Kandidater opstiller specifikt til Foreningsformand - og Forretningsrådsformandsposten og vælges direkte af Fællesrådet. Den valgte Foreningsformand og Forretningsrådsformand kan ikke være samme person. Foreningsformanden vælges før Forretningsrådsformanden.

\stk{3} De resterende bestyrelsesmedlemmer vælges af Fællesrådet blandt dertil opstillede kandidater. Disse medlemmer vælges uden post. Bestyrelsesmedlemmer skal være medlemmer af Polyteknisk Forening.

\stk{4} De valgte bestyrelsesmedlemmer fordeler posterne mellem sig i perioden frem til det konstituerende fællesrådsmøde. Den konstituerede bestyrelse skal som minimum dække følgende poster: en Foreningsformand, en Forretningsrådsformand, en Socialudvalgsformand, en S-husformand og en Uddannelsespolitisk Koordinator. Den kommende bestyrelse udarbejder et samlet visionsoplæg for foreningen. Visionsoplægget er et åbent dokument.

\stk{5} Fællesrådet vælger hvert medlem af bestyrelsen personligt og kan derfor til enhver tid afsætte enkelte medlemmer af bestyrelsen.

\item Bestyrelsen håndhæver Polyteknisk Forenings love og statutter og varetager den daglige ledelse. Foreningen tegnes juridisk af Formanden sammen med et andet medlem af bestyrelsen. I økonomiske og personalemæssige sammenhænge tegnes foreningen dog af Forretningsrådsformanden.

\item Bestyrelsen er ansvarlig for, at foreningen drives i overensstemmelse med de af Fællesrådet vedtagne retningslinjer. Bestyrelsen skal agere i forhold til de af Fællesrådet i principkataloget nedfældede principper.


%%%%%%%%%%%%%%%%%%%%%%%%%%%%%%%%%%

\subsection{Forretningsrådet}
\label{L:kap:FRR}
\item Fællesrådet vælger ved personvalg, på et ordinært fællesrådsmøde i april eller maj, de kommende forretningsrådsmedlemmer, der tiltræder 1. juni samme år. Dog undtagen forretningsrådsformanden.

\stk{2} Forretningsrådet består af forretningsrådsformand, 4 ordinære medlemmer samt 4 rådgivende medlemmer.

\stk{3} Kandidater opstiller specifikt til nedestående poster, og vælges direkte af fællesrådet. De ordinære medlemmer skal være eller havet været medlem af Polyteknisk Forening.

\begin{enumerate}
\item Scenelys
\item S-huset
\item Sekretariat
\item Forening
\end{enumerate}

\stk{4} De valgte medlemmer af forretningsrådet fordeler forretningsrådes øvrige opgaver mellem sig i perioden frem til de tiltræder den 1. juni.

\stk{5} De 4 rådgivende forretningsrådsmedlemmer vælges af Fællesrådet blandt dertil opstillede kandidater. Disse medlemmer vælges uden post.

\item Fællesrådet vælger hvert medlem af forretningsrådet personligt og kan derfor til enhver tid afsætte enkelte medlemmer af forretningsrådet.
\item Forretningsrådet er beslutningstagende instans i foreningens økonomiske dispositioner, og fører tilsyn med at disse træffes i overensstemmelse med de af Fællesrådet givne rammer. Forretningsrådet varetager desuden medarbejderledelse.
\item Forretningsrådet kan tillade at andre personer deltager i møderne.
\item Forretningsrådet kan meddele prokura.
\item I økonomiske og personalemæssige sammenhænge tegnes foreningen af Forretningsrådsformanden.
\item Forretningsrådet er ansvarlig for at der udføres revision af Polyteknisk Forenings regnskab jf. gældende lovgivning	

%%%%%%%%%%%%%%%%%%%%%%%%%%%%%%%%%%%%%%%%%%%%%

\subsection{Faste udvalg}
\label{L:kap:fasteUdvalg}
\item I tilknytning til de faglige råds og Fællesrådets arbejde skal der nedsættes en uddannelses politisk gruppe, et uddannelsespolitisk råd og et socialudvalg. Herudover kan Fællesrådet nedsætte aktivitetsudvalg for selvstændige faste aktiviteter som Fællesrådet måtte ønske.

%%%%%%%%%%%%%%%%%%%%%%%%%%%%%%%%%%%%%%%%%%%%%

\subsection{Uddannelsespolitisk Gruppe}
\label{:UPG}
\item Uddannelsespolitisk Gruppe har til opgave at udarbejde Polyteknisk Forenings uddannelsespolitik på nationalt plan og mandat til at repræsentere Polyteknisk Forening i disse spørgsmål.

\stk{2} Fællesrådet kan pålægge Uddannelsespolitisk Gruppe at arbejde med bestemte problemstillinger.

\stk{3} Uddannelsespolitisk Råd kan pålægge Uddannelsespolitisk Gruppe at arbejde med bestemte problemstillinger. Uddannelsespolitisk Gruppe er desuden forpligtet til at følge et mandat eller en principbeslutning fra Uddannelsespolitisk Råd.

\item Fællesrådet vælger ved personvalg, på et ordinært fællesrådsmøde i november eller december, de kommende medlemmer af Uddannelsespolitisk Gruppe, der skal tiltræde på det konstituerende fællesrådsmøde i februar.

\stk{2} Der vælges årligt 4 personer for en toårig periode. De valgte personer skal være medlem af Polyteknisk Forening.

\stk{3} Fællesrådet vælger ethvert medlem af Uddannelsespolitisk Gruppe personligt og kan derfor til enhver tid afsætte enkelte medlemmer.

\stk{4} Den af Fællesrådet valgte Uddannelsespolitisk Koordinator jvf. \lref{L:best:valg} er født formand.

\item Uddannelsespolitisk Gruppe fungerer som forretningsråd for Uddannelsespolitisk råd, og er ansvarlig for afholdelse af møder samt oplæring og opkvalificering af medlemmerne.


%%%%%%%%%%%%%%%%%%%%%%%%%%%%%%%%%%%%%%%%

\subsection{Uddannelsespolitisk Råd}
\label{L:kap:upr}
\item Uddannelsespolitisk Råd har til opgave at koordinere de studerendes repræsentanters arbejde i de styrende organer herunder DTU’s Bestyrelse, Akademisk Råd, tværgående uddannelsesudvalg og studienævn. I spørgsmål af principiel karakter forelægges disse for Fællesrådet til godkendelse.

\stk{2} Uddannelsespolitisk Råd har til opgave at udarbejde og vedtage Polyteknisk Forenings
uddannelsespolitik på institutionelt niveau efter indstilling fra Uddannelsespolitisk Gruppe.

\item Uddannelsespolitisk Råd består af alle de af Fællesrådet opstillede studenterrepræsentanter i DTU’s Bestyrelse, Akademisk Råd, tværgående uddannelsesudvalg og studienævn. Desuden er den af Fællesrådet valgte Uddannelsespolitiske Koordinator jvf. \lref{L:best:valg} formand.


%%%%%%%%%%%%%%%%%%%%%%%%%%%%%%%%%%%%%%%%

\subsection{Socialudvalget}
\label{L:kap:socialudvalg}
\item Socialudvalget har til opgave at arbejde for de studerendes sociale, boligmæssige og økonomiske interesser overfor DTU og samfundet.\\

\item Socialudvalget består af 2 repræsentanter for hvert af de faglige råd. Disse betegnes ordinære medlemmer. På det konstituerende møde kan der desuden optages interesserede, som ønsker at blive medlem. Desuden er Socialudvalgsformanden jvf. \lref{L:best:valg} medlem.

\item Socialudvalget vælger på sit konstituerende møde en næstformand.

\item Socialudvalget nedsætter på et årligt møde et indstillingsudvalg. Dette fungerer indtil et nyt er udpeget.

\stk{2} Indstillingsudvalget består af Socialudvalgsformanden, samt 9 personer udpeget på det årlige møde og 1 person udpeget af kollegianerforeningerne på de kollegier, som Polyteknisk Forening har indstillingsretten til, og skal godkendes af Fællesrådet.

\stk{3} Indstillingsudvalget varetager indstillingsarbejdet til værelser og lejligheder på de kollegier, som Polyteknisk Forening har indstillingsretten til. Indstillingsudvalget vælger blandt sin midte en formand.


%%%%%%%%%%%%%%%%%%%%%%%%%%%%%%%%%%%%%

\subsection{Aktivitetsudvalg}
\label{L:kap:aktivitetsudvalg}
\item Aktivitetsudvalg har til opgave at varetage en af foreningen prioriteret fast aktivitet indenfor foreningens formål.

\item \label{L:Aktivitet:nedsaettelse} Aktivitetsudvalg nedsættes af Fællesrådet efter indstilling fra Polyteknisk Forenings bestyrelse. Fællesrådet fastlægger de nærmere retningslinjer for aktivitetsudvalget via et af Fællesrådet vedtaget kommissorium og/eller et godkendt opstillingsgrundlag. Disse retningslinjer kan kun ændres af Fællesrådet. I retningslinjerne skal beskrives udvalgets formål og økonomi, herunder særskilt om hvorvidt Foreningen skal deltage i et evt. overskud eller underskud, samt om eventuel anden kapital skal tilfalde Foreningen eller andre institutioner el. lign.. Endelig kan der i retningslinjerne træffes afgørelser om anden procedure og valgform end angivet i statutterne.

\stk{2} Aktivitetsudvalgets medlemsliste og formandskandidat skal i forbindelse med det konstituerende Fællesrådsmøde forelægges Fællesrådet og godkendes her.

\item Formanden for hvert udvalg er ansvarlig overfor bestyrelsen og Fællesrådet for udvalgets økonomiske dispositioner, og for at medlemslisterne holdes ajour. Har et udvalg ikke en formand udpeges et medlem af udvalget til at bære dette ansvar.

%%%%%%%%%%%%%%%%%%%%%%%%%%%%%%%%%%%%%

\subsection{Studenterhuset}
\item Regler og retningslinjer for ledelse og administration af Studenterhuset, indenfor aftale mellem Undervisningsministeriet og Polyteknisk Forening om brugsretten til huset, fastsættes af Fællesrådet.


%%%%%%%%%%%%%%%%%%%%%%%%%%%%%%%%%%%%

\subsection{Grundfond} \label{L:kap:grundfond}
\item Foreningen har den 2013-01-01 omlagt foreningens grundfond således at grundfonden har en værdi af 5.000.000,00 kr. For foreningens grundfond skal følgende regler være gældende:
\begin{enumerate}
\item Grundfondens midler skal til enhver tid foreligge som kontant indlån i pengeinstitut og/eller anbragt i børsnoterede værdipapirer. Anbringes porteføljen i værdipapirer skal dette gøres gennem en af Finanstilsynet godkendt kapitalforvalter, hvor midlerne placeres i en veldiversificeret portefølje. Ved veldiversificeret forstås en kombination af
obligationer og aktier, hvor aktierne skal spredes geografisk og branchemæssigt. 
\item Ved den årlige regnskabsafslutning tillægges grundfonden et beløb således at formuens realværdi bibeholdes.
\begin{description}
\item[a] Set over en fem års periode skal grundfonden tilsikres at stige med minimum et beløb svarende til de fem foregående års inflation.
\item[b] Dette sikres ved en afkastkonto hvortil alt afkast og renter fra grundfonden tilskrives.
\item[c] Såfremt beløbet på afkastkontoen er større end det nødvendige beløb til sikringen af grundfonden, kan det overskydende beløb tilføres driften.
\end{description}
\item Fællesrådet kan beslutte at overføre yderligere midler til grundfonden med 2/3 flertal af de tilstedeværende stemmeberettigede.
\item Grundfondens midler kan kun anvendes såfremt og i det omfang, dette besluttes på to på hinanden følgende fællesrådsmøder af mindst 2/3 af tilstedeværende stemmeberettigede medlemmer, dog mindst halvdelen af samtlige stemmeberettigede. Det skal ske  med mindst én og højst tre måneders mellemrum.
\item Grundfondens investering og regulering af denne skal fremgå af foreningens årsregnskab. 
\end{enumerate}
\item Overgangsparagraf: Under omlægningen hæves dele af grundfonden og pengene benyttes til at tilbagebetale kassekreditten og sikre foreningens likviditet. Overgangsparagraffen udløber d. 16. januar 2013.

%%%%%%%%%%%%%%%%%%%%%%%%%%%%%%%%%%%%%%%%%%%%%%%%

\subsection{Lov- og statutændringer}
\item \label{L:LS:lovaedringer} Lov- og statutændringer kan kun ske efter, at Fællesrådet har haft nedsat et udvalg. Dette udvalg forelægger et evt. ændringsforslag. Forslaget skal udgøre et selvstændigt punkt på dagsordenen. Lovændringer kan kun ske, hvis ændringsforslaget vedtages enslydende på 2 ordinære Fællesrådsmøder af 2/3 af de tilstedeværende stemmeberettigede, dog mindst halvdelen af samtlige stemmeberettigede. Det skal ske med mindst en og højst tre måneders mellemrum. Statutændringer kan ske efter samme afstemningsregler, men er gyldige efter vedtagelse på et enkelt ordinært Fællesrådsmøde.

\item \label{L:LS:oploesning} Opløsning af Polyteknisk Forening kan alene ske efter reglerne i \lref{L:LS:lovaedringer} om lovændringer. I forslag om Foreningens opløsning skal indgå forslag om anvendelse af Foreningens midler (jvf. nedenfor). Såfremt opløsningen af Foreningen vedtages, forestås opløsningen af en eller flere af Fællesrådet valgte likvidatorer. Foreningens eventuelle formue skal tilfalde studentersociale formål for de studerende ved DTU. Såfremt der ikke kan udpeges en dertil egnet forening eller lignende, indgår midlerne - eventuelt foreløbigt - i en hertil indrettet fond, hvis ledelse består af 5 personer. Hvoraf DTU udpeger en og de ministerier der er relevante for de ingeniørstuderendes undervisning i samarbejde udpeger 1 person. De resterende 3 personer udpeges blandt de studerende på DTU af Fællesrådet til sidste møde. Ledelsen udarbejder selv de nødvendige statutter for fonden og træffer beslutninger om dennes virke herunder eventuel opløsning ved overførsel af midlerne til en eller flere andre organisationer, der kan opfylde de studentersociale formål.

\end{list}