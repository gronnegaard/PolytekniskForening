\setcounter{section}{19}
\setcounter{subsection}{0}
\section*{Statutter for Polyteknisk Forening}
\addcontentsline{toc}{section}{Statutter for Polyteknisk Forening}
%\subsection*{Kap. S.1}

\begin{list}
{\S\ S.\arabic{fpara}}{\usecounter{fpara}}
    \setlength{\labelwidth}{.5in}%
    \setlength{\leftmargin}{.25in} %

\subsection{Valg til Fællesrådet}
\label{kap:valgtilfaellesraad}
% \setcounter{fpara}{1}
%\ref{kap:valgtilfaellesraad}
%\nameref{kap:valgtilfaellesraad}

\item \label{S:FR:valgtilFR} Valget af de faglige råds repræsentanter i Fællesrådet foregår ved en afstemning ved simpel stemmeflerhed, idet hvert ordinært medlem af det faglige råd kan stemme på op til det antal repræsentanter jvf. \lref{L:FR:ValgtilFR}, som skal vælges.

\item \label{S:FR:nyvalgFR} Nedlægger et af rådenes Fællesrådsmedlemmer sit mandat, afholdes der partielt nyvalg til denne post. Dette valg foregår ved simpel stemmeflerhed blandt det faglige råds ordinære medlemmer.

\item \label{S:FR:omvalgFR} Såfremt der i et råd er utilfredshed med et af rådets Fællesrådsmedlemmer, kan vedkommende fratages sit mandat ved beslutning i det faglige råd. Mandatfratagelsen skal udgøre et punkt på dagsordenen for det faglige rådsmøde, og skal besluttes med 2/3 flertal. Kun de ordinære medlemmer af rådet kan deltage i denne afstemning, og mindst halvdelen af disse skal være tilstede, for at mandat-fratagelsen kan besluttes. Nyvalg til posten afholdes efter reglerne i \sref{S:FR:valgtilFR} .


%%%%%%%%%%%%%%%%%%%%%%%%%%%%%%%%%%%%%

\subsection{Valg til de faglige råd}
\label{S:kap:valgtilfagligraad}
\item \label{S:Faglig:ValgtilFaglig} Valg til de faglige råd skal ske hvert år i perioden 1. til 15. oktober, og mandaterne gælder fra og med de faglige råds konstituerende møder.\\
\\
\underline{Stk. 2} \\
Nye faglige råd kan konstitueres ved et ekstraordinært valg, så længe dette er annonceret 10 dage i forvejen via relevante kanaler, såsom foreningens hjemmeside og plakater.\\
\\
\underline{Stk. 3} \\
Nye medlemmer kan vælges og konstitueres til de faglige råd, så længe dette er annonceret 10 dage i forvejen via relevante kanaler, såsom foreningens hjemmeside og plakater.\\

\item \label{S:Faglig:Valgudvalg} Fællesrådet nedsætter senest 5 uger før valgene et valgudvalg og fastsætter de nøjagtige datoer for valgets afholdelse.\\
\\
\underline{Stk. 2}\\
Ved ekstraordinært valg jvf. \hyperref[S:Faglig:ValgtilFaglig]{\S \ S.\ref*{S:Faglig:ValgtilFaglig} Stk. 2} varetager FRFU funktionen som valgudvalg.\\

\item Valgudvalget forestår valgene til de faglige råd, og skal indeholde mindst et bestyrelsesmedlem samt 3 personer valgt af fællesrådet. Valgudvalget sørger, senest 3 uger før valgene, for offentliggørelse af tid, sted, samt opstillingsfrist for valget.

\item \label{S:Faglig:Blanket} Opstilling til valgene til de faglige råd skal ske senest på valgdagen og sker ved udfyldelse af en opstillingsblanket indeholdende navn, studienummer, adresse og underskrift. Blanketten skal forefindes mindst 3 uger før valget.

\item Hvis et medlem udtræder i utide, kan Fællesrådet bestemme, at der skal afholdes partielt nyvalg. Fællesrådet fastsætter de nærmere regler, idet \sref{S:Faglig:Valgudvalg} til \sref{S:Faglig:Blanket} skal overholdes.\\

\item Ekstraordinært valg holdes for alle eksisterende råd, når det vedtages på et Fællesrådsmøde med mindst 2/3 flertal, dog mindst halvdelen af samtlige stemmeberettigede. Ekstraordinære valg afholdes efter reglerne i \sref{S:Faglig:Valgudvalg} til \sref{S:Faglig:Blanket}.

\item Protester over et afholdt valg skal fremsættes skriftligt overfor bestyrelsen senest en uge efter valgets afholdelse. Fællesrådet behandler evt. indkomne klager og tager stilling til evt. afholdelse af omvalg efter reglerne i \sref{S:Faglig:Valgudvalg} til \sref{S:Faglig:Blanket}.


\subsection{Valg til bestyrelsen} \label{kap:ValgTilBestyrelsen}
\item \label{S:bestyrelsen:BXX-eu valg} For at forberede valget af det kommende års bestyrelse nedsætter Fællesrådet mindst fem måneder før det konstituerende fællesrådsmøde et bestyrelsens-evalueringsudvalg. Udvalget kan inddrage andre interesserede i sit arbejde og refererer løbende til Fællesrådet. I efteråret skal der før personvalget afholdes mindst fem opkvalificerende møder om generelle emner. Disse skal promoveres bredt på DTU. Møderne er åbne for alle studerende ved DTU.

\item Bestyrelsesmedlemmerne vælges på baggrund af skriftlig motivation og CV. Begge dele er åbne dokumenter.\\
\\
\underline{Stk. 2}\\
Foreningsformandens og Forretningsrådsformandens arbejde defineres ved minimumskrav i principkataloget. Herudover kan de selv definere mål og arbejdsopgaver.\\

\item Hvis hele bestyrelsen afgår på én gang skal Fællesrådets Forretningsudvalg (FRFU) indkalde til fællesrådsmøde, der skal finde sted senest 24 timer efter bestyrelsens afgang. På dette fællesrådsmøde skal der vælges en ny bestyrelse. Indtil fællesrådsmødet ligger PF's juridiske og økonomiske forpligtelser i FRFU.\\
\\
\underline{Stk. 2}\\
Hvis enkelte bestyrelsesmedlemmer vælger at forlade deres post, skal der vælges en ny i stedet. Dette skal ske ved førstkommende fællesrådsmøde, indtil da er det op til den siddende bestyrelse at varetage den afgåedes opgaver.\\
\\
\underline{Stk. 3}\\
Fællesrådet kan afsætte bestyrelsen. Dette skal ske med mindst to tredjedeles flertal, og mindst halvdelen af de ordinære medlemmer skal være til stede. Herefter konstitueres en bestyrelse, der fungerer indtil Fællesrådet vælger en ny bestyrelse.

\subsection{Valg til faste udvalg og aktivitetsudvalg} \label{kap:ValgTilUdvalg}
\item Såfremt intet andet er fastlagt jvf. \lref{L:Aktivitet:nedsaettelse} sker valg til faste udvalg og aktivitetsudvalg efter procedure beskrevet i efterfølgende statutter.\\

\item Aktivitetsudvalg består af alle interesserede, der på det konstituerende møde ønsker at blive medlem. De faste udvalg og aktivitetsudvalgene afholder konstituerende møder i løbet af de sidste 3 uger op til det konstituerende Fællesrådsmøde. Dog kan fællesrådet beslutte at enkelte aktivitetsudvalg godkendes som gruppe på grundlag af opstillingsgrundlag, fællesrådet kan samtidig hvis de finder det formålstjenstligt for udvalgets arbejde, beslutte at nedsætte dette på andre tidspunkter end de ovenfor nævnte. Et aktivitetsudvalg indstiller på sit konstituerende møde et forslag til formand, såfremt en sådan er defineret, som sammen med medlemslisten forelægges Fællesrådet til godkendelse. Formanden skal være studerende ved DTU. Forkastes den indstillede person af Fællesrådet indstiller aktivitetsudvalget 3 personer, iblandt hvilke Fællesrådet skal foretage det endelige valg. Den først afviste person må ikke være mellem disse 3. Faste udvalg skal på konstituerende møde besætte statutfæstede poster.

\item  Såfremt et medlem af et aktivitetsudvalg udebliver fra to møder uden afbud, kan udvalget eller udvalgets eventuelle formand slette vedkommende fra udvalgets medlemsliste.


\subsection{Fællesrådsmøder}
\label{S:kap:FRmoeder}
\item Der afholdes med passende mellemrum ordinære Fællesrådsmøder dog mindst 8 om året, hvoraf et er konstituerende, og et er afsluttende Det konstituerende Fællesrådsmøde afholdes i løbet af de første syv dage af forårssemesterets 13-ugersperiode.

\item  Ved det konstituerende eller det efterfølgende møde nedsætter fællesrådet et forretningsudvalg og vedtager hvilke opgaver og kompetencer, jvf. dog \sref{S:FRmoeder:indkaldelse} og \sref{S:FRmoeder:ekstra}, udvalget har.

\item  \label{S:FRmoeder:indkaldelse} Ordinære Fællesrådsmøder indkaldes skriftligt af fællesrådets forretningsudvalg senest otte dage forud med en foreløbig dagsorden som skal offentliggøres i foreningens hjemmeside, således at alle studerende har let og ubesværet adgang til denne, senest en uge før mødet. Senest tre dage før mødet kan påføres punkter, som ønskes behandlet, og herefter udsender fællesrådets forretningsudvalg den endelige dagsorden hvis der er ændringer. Ethvert Fællesrådsmedlem kan kræve punkter sat på Fællesrådets dagsorden. Et punkt kan ved mødets start komme på dagsordenen hvis 2/3 af de tilstedeværende enes herom, og mindst halvdelen af samtlige stemmeberettigede er tilstede.                                     
                             
\item \label{S:FRmoeder:ekstra} Ekstraordinære fællesrådsmøder kan i hastetilfælde indkaldes af bestyrelsen eller Fællesrådets forretningsudvalg med et døgns varsel, idet hvert medlem af Fællesrådet så vidt muligt skal modtage besked om mødet med angivelse af dagsorden inden denne frist. Desuden kan ethvert ordinært Fællesrådsmedlem indkalde til ekstraordinært Fællesrådsmøde, såfremt mindst 2/3 af samtlige ordinære fællesrådsmedlemmer skriftligt støtter dette.

\item  Et valgt medlem af Fællesrådet kan før et møde ved skriftlig meddelelse til dirigenten overdrage sin stemmeret til et andet navngivet medlem af samme faglige råd. Såfremt et ordinært Fællesrådsmedlem ikke er tilstede kan det pågældende faglige råds suppleant uden fuldmagt overtage den fraværendes stemmeret. Stemmeret kan ikke overtages eller overdrages under et igangværende møde.

\item  Alle råd, udvalg og bestyrelsen aflægger mindst en gang om året en samlet skriftlig beretning, som opsummerer og vurderer det forgangne års arbejde. Beretningen skal desuden lægge op til en diskussion af det kommende års arbejde. Alle beretninger skal behandles på det afsluttende Fællesrådsmøde, hvortil de derfor skal foreligge.


\subsection{Møder i de faglige råd}
\label{S:kap:FagligMoeder}
\item De faglige råd holder møder løbende efter behov dog mindst 6 om året. De faglige råds konstituerende møder skal afholdes senest en måned efter valgresultatet fra valg til de faglige råd er offentliggjort.

\item  \label{S:FagligMoeder:indkaldelse} Ordinære møder i de faglige råd indkaldes skriftligt på foreningens hjemmeside, således at den er synlig for alle studerende, på relevante retninger, og at alle har let og ubesværet adgang til denne og ved opslag af rådets forretningsudvalg mindst 5 dage før mødets afholdelse med angivelse af dagsorden. 

\item  Ekstraordinære møder i de faglige råd kan kræves indkaldt af 1/4 af rådet. Herefter indkaldes der efter reglerne for ordinære møder i \sref{S:FagligMoeder:indkaldelse}.

\item  Et fagligt råd kan til enhver tid give enhver studerende ved DTU ret til at deltage i møderne. De ordinære medlemmer kan endvidere tildele disse studerende stemmeret for en kortere eller længere periode, indtil det pågældende råds årsafslutning. Disse rettigheder kan fratages de pågældende efter udeblivelse fra to møder uden afbud. Det er ikke muligt at opnå stemmeret i to eller flere faglige råd. Stemmeret kan ikke overdrages under et møde.


\subsection{Møder generelt}
\label{S:kap:MoederGenerelt}
\item  Stemmeret kan udøves af de tilstedeværende medlemmer. Ingen kan have mere end én stemme.

\item \label{S:MoederGenerelt:fortrolig} De faglige råd og Fællesrådet skal under drøftelse af enkelte spørgsmål af særlig fortrolig karakter (f.eks. personsager, finansielle dispositioner, valgtaktiske spørgsmål), udelukke ikke-ordinære medlemmer fra mødet, såfremt mindst 2 ordinære medlemmer måtte ønske dette. Rådene kan endvidere vedtage at et helt møde, skal betragtes som fortroligt. Ikke-ordinære medlemmer, der ved deres adfærd forstyrrer arbejdet, kan efter forudgående varsel udvises.

\item  Fællesrådsmøder og møder i de faglige råd ledes af en af rådet udpeget dirigent. Såfremt der ytres mistillid til dirigenten, sættes denne til afstemning, og dirigenten kan væltes med simpelt flertal. Fællesrådet kan stille krav til procedurer for gennemførelse af møder i Fællesråd, faglige råd og øvrige møder i Polyteknisk Forening.

\item  Ved Fællesrådsmøder og møder i de faglige råd udpeges der en referent, som er ansvarlig for udfærdigelsen af referatet. Det fuldstændige referat skal altid være at finde i Foreningens arkiv. Referaterne udsendes til medlemmerne senest tre uger efter mødet og foreligger til godkendelse på næste møde. Referater af Fællesrådsmøder, med undtagelse af fortrolige punkter, skal offentliggøres således at alle studerende har let og ubesværet adgang til disse og som minimum på foreningens hjemmeside.

\item  En forsamling er kun beslutningsdygtig, såfremt mindst 1/3 af forsamlingens ordinære stemmeberettigede medlemmer er til stede dog minimum 7 eller samtlige medlemmer. Er forsamlingen ikke beslutningsdygtig, kan der med mindst 5 dages varsel indkaldes til et nyt møde med samme dagsorden. Denne forsamling er, uanset deltagerantallet, beslutningsdygtig.

\item  Afstemning foregår ved håndsoprækkelse eller hemmeligt, såfremt mindst 2 stemmeberettigede begærer dette. Når ikke andet er angivet, afgøres disse ved simpelt flertal. Ved stemmelighed bortfalder det stillede forslag. Ved stemmelighed til personvalg foretager dirigenten lodtrækning.

\item  Ved ethvert møde skal udvalg, undergrupper, repræsentanter, bestyrelse og andre fortælle om deres arbejde siden sidste møde.

\item  Medlemmer af bestyrelsen har ret til at deltage i alle møder i Foreningen.


\subsection{Økonomi}
\label{S:Kap:oekonomi}
\item  Foreningens regnskabsår er fra 1. juni til 31. maj. Bestyrelsen opstiller et budget for året og forelægger det til godkendelse på et Fællesrådsmøde inden udgangen af september.

\item  Årsregnskab skal godkendes af Fællesrådet inden 1. december i det følgende budgetår. Bestyrelsen sørger som minimum for, at det samlede årsregnskab udsendes til Fællesrådet mindst 1 uge før det Fællesrådsmøde, hvor disse skal behandles. Samtidig med udsendelse til Fællesråds-medlemmerne sørger bestyrelsen for, at regnskaberne for studenterpolitiske aktiviteter, sociale aktiviteter og Studenterhuset fremlægges på PF sekretariatet.

\item  Kritisk revisions beretning behandles på det førstkommende ordinære fællesrådsmøde efter det konstituerende fællesrådsmøde. Bestyrelsen sørger for de kritiske revisorers beretning udsendes til Fællesrådet mindst 1 uge før dette Fællesrådsmøde.

\item \label{S:okonomi:standardregler} Fællesrådet kan vedtage standardregler om økonomi og administration for klubber m.v. tilknyttet Polyteknisk Forening evt. efter henstilling fra Klubudvalget.