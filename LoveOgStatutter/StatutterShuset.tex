%\setcounter{section}{18}
\section*{Statutter for S-huset}
\addcontentsline{toc}{section}{Statutter for S-huset}
\subsection{Kap. S.9 Klubudvalget}
\label{s:klubudvalget}
\begin{list}
\item Klubudvalget er det rådgivende organ vedrørende klubberne under PF. Her Fra kommer der indstillinger til den siddende bestyrelse om tildeling/fratagelse af lokaler, oprettelse/nedlæggelse af klubber, samt de daglige retningslinier for klubberne. Er der tvivlsspørgsmål fra klubberne, er det Klubudvalget der skal kontaktes, mens de endelige afgørelser træffes af den siddende bestyrelse for foreningen. Klubudvalget er ansvarlig for afholdelsen af PF åbenthus der er sammenfaldende med DTU åbent hus der afholdes to gange årligt.
\item Klubudvalget består af:
	\begin{enumerate}
	\item[•] To repræsentanter fra PF’s bestyrelse hvoraf den ene er født formand.
	\item[•] Et medlem af Fællesrådet udpeget af Fællesrådet.
	\item[•] CREW-formanden for s-huset.
	\item[•] Et medlem fra hver klub i foreningen.
	\end{enumerate}
\item Klubudvalget godkendes af Fællesrådet på det konstituerende fællesrådsmøde.
\item Klubudvalgsmøder indkaldes af Klubudvalgsformanden. Alle medlemmer kan kræve punkter sat på dagsordenen. Der skal indkaldes til møde, såfremt ét medlem kræver det. Der afholdes mindst 4 møder om året. Det konstituerende møde af Klubudvalget afholdes senest en måned efter fællesrådsmødets godkendelse af Klubudvalget.
\item Indkaldelse til Klubudvalgsmøde skal ske skriftligt til medlemmerne og på foreningens hjemmeside, senest 5 dage før mødets afholdelse, med angivelse af dagsorden.
\item PF's procedureregler gælder for møder i Klubudvalget.
\item Klubudvalgsformanden har pligt til at indsamle indstillinger fra klubber om repræsentation i Klubudvalget inden det konstituerende fællesrådsmøde. Hver klub, som er godkendt som klub under Polyteknisk Forening, kan give én indstilling på én person.
\item Godkendte klubber er underlagt Klubudvalget.
\item Klubudvalget sørger for, at klubberne mindst én gang om året deltager i PF Åbent hus, hvor ikke-medlemmer har mulighed for at stifte bekendtskab med klubberne.\\
\\
%\textit{Godkendt af Fællesrådet på FR158 d. 1/11 2007}
\subsection{Kap. S.10 S-husets Aktivitetsudvalg}
\label{s:shus-aktivitetsudvalg}
Aktivitetsudvalget står for planlægning af aktiviteterne i S-Huset. For at gøre dette skal der afholdes et
aktivitetsudvalgsmøde før semesterstart, for at kunne lægge programmet for semestret:
-Musikprogrammet planlægges af Polyjoint Booking.
-PR-udvalget planlægger reklame i forbindelse med alle arrangementer, herunder Fredagsrock, Joints,
turneringer m.m.
\item
Aktivitetsudvalget består af:
	\begin{enumerate}
	\item[•] S-Husformanden
	\item[•] Crewformanden
	\item[•] To S-Husets crew valgt af de resterende medlemmer af S-husets crew på det første crewmøde i
	semestret
	\item[•] En repræsentant fra Polyjoint Booking
	\item[•] En repræsentant fra Polyteknisk Scenelys
	\item[•] En af Indkørerne
	\item[•] En repræsentant fra S-Husets PR-udvalget
	\item[•] Bestyreren
	\end{enumerate}
Aktivitetsudvalget godkendes på det konstituerende Fællesrådsmøde.
\item Der afholdes åbne møder før hvert større arrangement.
\item Møder i Aktivitetsudvalget indkaldes af S-Husformanden. Alle medlemmer kan kræve punkter sat på dagsorden. Der skal indkaldes til møde, såfremt to medlemmer ønsker det.
\item Aktivitetsudvalget er ansvarlig for der udarbejdes en evaluering efter hvert arrangement.
\end{list}
