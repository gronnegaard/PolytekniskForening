\section{Forretningsorden for Fællesrådet}

\subsubsection{Mødefrekvens og -indkaldelse}
Der afholdes minimum 8 ordinære fællesrådsmøder i løbet af kalenderåret. En mødekalender, indeholdende fordeling af mad-, kage- og oprydningstjanser, vedtages af fællesrådet. For forårssemesteret vedtages kalenderen på det konstituerende møde. For efterårssemestret vedtages kalenderen på det sidste møde i forårssemestret. Såfremt et råd ikke har mulighed for at påtage sig den angivne tjans, er det rådets eget ansvar at sørge for at der er andre der påtager sig denne tjans. FRFU sørger for at udsende en mødeindkaldelse til fællesrådet hvor man kan tilmelde sig mad.\\
\\
FRFU udsender senest 10 dage før ordinære møder en mødeindkaldelse indeholdende en dagsorden samt eventuelle åbne bilag. Forslag til dagsorden og bilag skal derfor være FRFU i hænde, så FRFU kan overholde denne tidsfrist. FRFU udsender senest 3 dage før ordinære møder den endelige dagsorden samt åbne bilag. FRFU skal sørge for at indkaldelser og åbne bilag er tilgængelige på foreningens hjemmeside. Såfremt der er lukkede bilag til mødet, kan disse afhentes på PF-sekretariatet, så snart indkaldelsen er udsendt.
%Såfremt punkter eller bilag ikke tilsendes FRFU rettidigt, kan FRFU, hvis det vurderes at punktet bør diskuteres i de faglige råd, udskyde punktet til et senere møde. Der henstilles til at ingen misbruger muligheden for at få punkter igennem ved at sætte disse på tre dage før afholdelsen af mødet, således at det ikke er muligt at få punktet behandlet i de faglige råd.
\subsubsection{Mødeafholdelse}
Referater udformes i henhold til de af Fællesrådet vedtagne retningslinier se \nameref{RetningslinjerForRefFR}. Dagsordenen kan ændres ved mødets start, såfremt mindst 2/3 af de stemmeberettigede er til stede og et flertal stemmer for ændringen, jvf. \sref{S:FRmoeder:indkaldelse}.\\
\\
Overdragelse af stemmeret sker på den af FRFU udsendte formular og afleveres til dirigentbordet før mødets start, eller sendes i scannet version pr. mail. Afbud til et møde, meldes via den fra FRFU’s side udsendte mail.\\
\\
FRFU indstiller dirigent(er) og referent(er) til møderne, der vælges af FR ved mødets start. Dirigenten(erne) bestemmer suverænt hvem der har taleret. FR kan til enhver tid under mødet vælge en anden dirigent eller referent, såfremt et flertal af FR ønsker dette. FRFU styrer en talerliste, der kan lukkes af dirigenten med forudgående varsel. Herefter er talerlisten lukket og åbnes først igen, når dirigenten(erne) meddeler herom.\\
\\
Dirigenten(erne) sørger for at holde pauser med passende mellemrum, løbende igennem mødet. Nogle pauser vil blive brugt til nærmere diskussion af punkter, såkaldte summepauser. Under summepauser forlader mødedeltagerne normalt ikke mødelokalet. Såfremt mindst to deltagere ønsker det, kan dirigenten beslutte at holde timeout, hvor de pågældende deltagere kan forlade lokalet og diskutere punktet. Andre pauser afholdes med henblik på toiletbesøg og forfriskninger. Forlader en mødedeltager mødet udenfor en pause, skal denne meddele dette til referenten og ligeledes melde sin tilbagevenden.

\subsubsection{Afstemninger}
Punkter afholdes normalt som åbne punkter. Hvis mindst to stemmeberettige ønsker det, skal punktet lukkes jvf. \sref{S:MoederGenerelt:fortrolig}.\\

Afstemninger afholdes normalt som åbne afstemninger med håndsoprækning. Hvis mindst to stemmeberettigede ønsker det, skal der laves en lukket afstemning, jf. love og statutter. Ved ændringsforslag stemmes der først om det mest yderligtgående forslag. Falder dette, stemmes der om det næste forslag og så videre. Falder alle forslag bibeholdes det oprindelige. Ved stemmelighed bortfalder ændringsforslaget. Ved personvalg stemmes der med én stemme pr post der skal besættes. Herefter er de personer med flest stemmer valgt. Ved stemmelighed foretager dirigentbordet lodtrækning.