\section{Forretningssektoren}
Pr. 31. februar 2012 er følgende forretningsforetagender kontrolleret af eller under indflydelse af Polyteknisk Forening: Polyteknisk Forenings Studentersociale Fond, Polyteknisk Boghandel, S-huset, Polyteknisk Forenings Sekretariat, Kollegiekonsulentordningen, Polyteknisk Forenings Instillingsudvalg, BEST og IAESTE.\\

Traditionelt er disse virksomheder i vid udstrækning selvstyrende. Formelt er forretningssektoren Fællesrådets ansvar og Forretningsrådets indstillinger er vejledende, men Fællesrådet bør altid overveje betydningen af Forretningsrådets kompetence og økonomiske indsigt.\\

\textit{Polyteknisk Forening skal som helhed sikre konsolidering af egenkapitalen således at foreningens fremtidige økonomiske muligheder ikke forringes.}\\

For at imødekomme politikken formuleret ovenstående har Fællesrådet truffet principbeslutning om en \textit{gennemsnitlig minimum} forrentning på 4\% af egenkapitalen.\\

Budgettet søges overholdt. Ved udsigt til budgetoverskridelse søges Fællesrådet om budgetændring.\\

Forretningsrådet rådføres vedrørende budget, regnskab, personalesager, egenkapitalsforvaltning og organisatoriske
ændringer.

\subsubsection{Kommercielle sektor}
\begin{itemize}
\item Gebyrer og brugerbetaling bør fastsættes under hensyntagen til foreningens økonomi samt under hensyntagen til de studerendes økonomiske forhold.
\item Profitmaksimering er ikke i modstrid med foreningens formål, så længe hensynet til de studerende prioriteres. Polyteknisk Forenings forretningsforetagender bør have øjnene åbne for nye indtjeningsmuligheder så længe deres hovedformål ikke nedprioriteres.
\item Forretningsforetagenderne bør have frie hænder til at iværksætte tiltag, såfremt de finder disse indenfor deres kompetenceområde, og såfremt bestyrelsen, forretningsrådsformanden og gerne Forretningsrådet har givet deres samtykke.
\item Lederne af Polyteknisk Forenings forretningsforetagender har ansvar for deres daglige drift og for imødekommelse af foreningens krav til disse. Det er forretningsrådsformandens opgave at holde kontakt til lederne.
\item Fællesrådet skal informeres ved uddeling af midler fra den studentersociale fond.
\end{itemize}

\subsubsection{Servicesektor}
\begin{itemize}
\item Polyteknisk Forenings service-organer yder en god service overfor de studerende og virker som aflastning for bestyrelsen og aktivisterne.
\item Det tilstræbes, at der ikke tjenes penge på obligatoriske lærebøger ved salg til studerende.
\end{itemize}

\subsubsection{Forbrugssektor}
\begin{itemize}
\item  Hvert år råder Polyteknisk Forening over en pulje til studentersociale formål. Puljens størrelse afhænger af årets resultat i Polyteknisk Forenings indtægtsgivende aktiviteter.
\item Alle med et studentersocialt formål kan søge om tilskud fra denne pulje.
\item Polyteknisk Forening forbeholder sig ret til at stille modkrav om regnskabsaflæggelse.
\item På baggrund af en afvejning af indkomne ansøgninger og foreningens øvrige økonomiske forpligtelser, fastlægges puljens fordeling af FR ved den årlige budgetlægning.
\item Anvendes de ansøgte penge ikke i overensstemmelse med ansøgningens ordlyd eller foreningens principper forbeholder Polyteknisk Forening sig ret til at træffe passende foranstaltninger.
\end{itemize}

\subsubsection{Sponsorering}
Polyteknisk Forening, dens arrangementer, lokaler, værdier, udvalg, samt alle afledninger heraf, må ikke sponsoreres af virksomheder, personer eller organisationer, såfremt:
\begin{itemize}
\item Sponsor konkurrerer med foreningen på ét eller flere områder, det være sig forretningsmæssigt eller andet. Sponsor er kontroversiel på en sådan måde, at dens værdier ikke kan forenes med det, som Polyteknisk Forening står for.
\item Sponsor kræver, eller hvis sponsoratet på en anden måde vil medføre, at Polyteknisk Forening ikke fremstår som suveræn arrangør eller indehaver af det sponsorerede arrangement eller de sponserede værdier.
\item Sponsoratet påfører foreningen sekundære udgifter.
\end{itemize}

Såfremt ovenstående er overholdt, vil foreningen tillade sponsorater, dog skal alle sponsorater godkendes af bestyrelsen. Man skal huske at melde tilbage efter modtagelsen af et sponsorat, så man vedligeholder den gode kontakt til virksomhederne.

\subsubsection{De faglige råds rådighedsbeløb}
\begin{itemize}
\item Rådighedsbeløbet er en pulje der kan ansøges til gavn for en hel retning.
\item Det er de enkelte faglige råd, som beslutter, hvad beløbet skal benyttes til, inden for de ansøgte rammer.
\item Rådighedsbeløbet må kun benyttes til alkohol i forbindelse med større sociale arrangementer dækkende hele retningen.
\begin{itemize}
\item Et fagligt rådsmøde anses ikke for værende et større socialt arrangement.
\end{itemize}
\item Bevilget beløb udbetales udelukkende ved forelæggelse af kvitteringer. Såfremt hele det bevilligede beløb ikke benyttes vil det resterende beløb forblive i puljen til ansøgelse.
\item Hvis et råd ønsker at indkøbe ting til gavn for retningen, som overstiger rådighedsbeløbets størrelse, kan rådet optage et rentefrit lån i Polyteknisk Forening, såfremt der økonomisk råderum til dette. Dette lån skal dog først godkendes af Polyteknisk Forenings bestyrelse.
\end{itemize}

\subsubsection{Logopolitik}
\begin{itemize}
\item Alle organer der trykker/præger materialer for økonomiske midler tilvejebragt af Polyteknisk Forening, skal samtidig sikre at der trykkes/præges et PF-logo på mediet.
\end{itemize}