\section{Politikpapir}

\subsection{Politikpapir omkring eksamensevaluering}
\subsubsection{Eksamensevaluering på DTU}
Dette papir definerer Polyteknisk Forenings holdning til eksamensevaluering på DTU.
På nuværende tidspunkt, år 2011, er det ikke muligt for studerende at evaluere deres eksaminer, selvom eksamensevaluering er den bedste mulighed for at give undervisere feedback på eksamen samt give studienævn mulighed for at evaluere kurser i deres helhed. Eksamensevaluering vil også kunne vise om eksamensformen og -niveauet er hensigtsmæssige i forhold til at vurdere om læringsmålene i kurset er blevet opfyldt.\\
\\
\underline{Polyteknisk Forening ønsker eksamensevaluering på alle kurser.}

\subsubsection{Eksamensevalueringens indhold}
Eksamensevaluering bør være et redskab til forbedring af undervisningen og bør derfor have et format som understøtter dette. Skemaet bør være tilgængeligt umiddelbart efter eksamenen og antallet af spørgsmål bør holdes på et minimum så flest mulige studerende besvarer alle spørgsmål. Der skal være en rubrik til kvalitative kommentarer, da erfaring har vist at det er denne del af evalueringer som er mest brugbare til udarbejdning af forbedringer. Yderligere bør der i skemaet henvises til læringsmålene for det enkelte kurser så alle studerende har fuldt kendskab til disse.\\
\\
\underline{Polyteknisk Forening ønsker at evalueringen sikrer en sammenligning af de givne læringsmål}
\underline{og det eksaminerede.}

\subsubsection{Mundtlig og skriftlig eksamen}
Der bør ikke være nogen forskel i den principielle udformning af evalueringen for mundtlige og skriftlige eksaminer idet begge eksamensformer er lige gyldige. Ved mundtlige eksaminer kan evalueringen bære præg af den givne karakter, men det forventes at studienævnene er kompetente i deres arbejde og forstår at uddrage det relevante.\\
\\
\underline{Polyteknisk Forening ønsker at alle evalueringer bliver behandlet ligeligt.}

\subsection{Fusionspolitik for PF}
Denne politik har til formål at klargøre en række problemstillinger i den aktuelle debat om universitetsfusioner i
Danmark med hensyn til DTU’s nuværende ingeniøruddannelser.\\

Polyteknisk Forening har igennem en længere periode beskæftiget sig med spørgsmål om eliteuniversiteter, uddannelse i verdensklasse og sikring af ingeniørprofilen. En række af vores holdninger til disse emner vil derfor naturligt indgå i denne politik, som herudover er baseret på temadrøftelser om universitetsfusioner afholdt i foreningen.\\

Det naturlige udgangspunkt for en universitetsfusion er at skabe en institution med nye muligheder, som fremover bidrager til at sikre Danmarks position som højteknologisk videnssamfund. En fusion, som DTU skal indgå i, må overordnet basere sig på værdier og mål, der i dag er kendetegnende for DTU. DTU bør fusionere med parter, der har et klart erhvervsmæssigt sigte og anvender teknisk og naturvidenskabelig viden til gavn for samfundet.\\

PF er positiv over for tiltag, som kan højne kvaliteten af ingeniøruddannelserne. En velgennemført og fremtidssikret fusion kan have denne effekt, hvorfor PF hilser en sådan udvikling velkommen. Der er dog samtidigt en risiko for at kvaliteten falder, hvorfor det er nødvendigt at have fokus rettet mod uddannelseskvaliteten under en fusion. PF mener, det er vigtigt, at både studerende og ansatte inddrages i alle aspekter af en kommende fusion for at opnå størst mulig synergieffekt. I PF ønsker vi derfor at deltage aktivt i processen for at sikre, at de bedste resultater opnås.
\subsubsection{Ingeniørfaglighed}
Danmark skal udfylde sin rolle i det globale billede ved at udvikle sig til et innovativt og forskningsbaseret videnssamfund med fokus på løsninger. En forudsætning for denne udvikling er en stærk ingeniøruddannelse. En universitetsfusion med DTU skal sikre en styrkelse af ingeniørprofilen. Herunder er det vigtigt at sikre den faglige identitet, de pædagogiske principper og den forskningsbaserede undervisning.\\

Den faglige identitet sikres blandt andet ved at fremhæve og dyrke forskellene mellem ingeniøruddannelserne og andre uddannelser, samt ved at holde fast på en institutstruktur med plads til ingeniørfaglige institutter. Det er vigtigt, at de pædagogiske principper, deriblandt den store mængde konfrontationstid, åben dør princippet (at underviseren kan kontaktes uden for undervisningstiden), samt det nære studiemiljø, bibeholdes efter en fusion. De studerende skal opleve det tætte samspil med undervisere og deres forskningsmiljø, som bidrager til øget engagement og læring. Institutstruktur på DTU bestående af mindre decentrale enheder kan med fordel bibeholdes, hvilket kan være med til at sikre det nære studiemiljø. Institutternes faglige sammensætning skal naturligvis genovervejes, og med en positiv og konstruktiv indstilling vil der kunne skabes nye faggrænser og muligheder. Med den skitserede struktur undgås en model med ét institut, én uddannelse, hvor tværfagligheden mistes.\\

Den forskningsbaserede undervisning og høje spidskompetence, som til dels erhverves gennem det afsluttende kandidatprojekt, med udgangspunkt i aktuel forskning, sikrer den studerendes akademiske niveau. Parallelt hermed tillægges uddannelsen en valgfrihed og fleksibilitet, der er afgørende for den enkelte studerendes engagement, selvstændiggørelse og medindflydelse på studiet. Disse principper bør fortsat være grundelementer i ingeniøruddannelsen i en fremtidig uddannelsesstruktur.\\

I forbindelse med fusionen er det vigtigt, at diplomingeniøruddannelsen bibeholdes som en selvstændig uddannelse med et tydeligt erhvervsrettet sigte. Diplomingeniøruddannelsen fra DTU, med udgangspunkt i CDIO-konceptet, bliver resultatet af de åbenlyse synergier, der opstår, når forskning, udvikling og anvendelse kombineres.DTU’s eksisterende diplomuddannelse skiller sig ud fra Ingeniørhøjskolernes uddannelser ved at opfylde et behov for diplomingeniører med et kendskab til forskningsmiljøet.

\subsubsection{Forskning og sektorforskning}
Forskningsmiljøet på en større og mere slagkraftig institution vil stå stærkere i forhold til det internationale forskningsmiljø i kraft af både bredde og spidskompetence. Denne situation vil være yderst gavnlig for både forskningen og uddannelsen, men det er vigtigt at holde sig for øje, at klassiske ingeniørgrene ikke overses i konkurrencen med nye tværfaglige miljøer. Der skal fortsat forskes og uddannes i teknisk-naturvidenskabelig grundfaglighed på en ingeniøruddannelsesinstitution.\\

PF ser store fordele i integrationen af sektorforskningsinstitutioner på universiteterne. Tilknytningen af disse institutioner vil kunne fremme både forskningen i form af øget samarbejde og uddannelserne i form af et større samlet forskningsområde. En del af den forskning, der sker på sektorforskningsinstitutionerne er mere anvendelsesorienteret og vil på den måde kunne styrke diplomingeniøruddannelserne yderligere. \\

Sammenlægningerne skal ikke blot forbedre DTUs forskningsstatistikker og øge antallet af forskere per studerende. For at en fusion med sektorforskningsinstitutioner skal kunne løfte ingeniøruddannelsen, er det vigtigt, at der sker et reelt øget samarbejde.For at sikre en god vidensdeling og formidling af forskningen skal alle universitetsforskere i større eller mindre grad undervise. Dette vil blandt andet kræve, at nogle forskere didaktisk og pædagogisk opkvalificeres, hvilket også vil være en yderligere fordel for uddannelserne.

\subsubsection{Økonomi}
En forudsætning for at undgå en kvalitetssænkning af ingeniøruddannelsen er, at de nuværende økonomiske rammer omkring uddannelsen som minimum fastholdes. En fusion vil naturligvis medføre øgede omkostninger i implementeringsfasen (øget administration og VIP ressourcer). Disse omkostninger bør afholdes af eksterne offentlige parter, så det ikke forringer undervisningskvaliteten under implementeringen eller overbebyrder personalet. Det er under implementeringsfasen statens ansvar, at der er økonomiske rammer til at opretholde universiteternes nuværende kvalitet.\\

Blandt de institutioner, der er på tale som fusionspartnere, befinder flere sig i en noget anderledes økonomisk situation end DTU. Tilføres der ikke yderligere midler i forbindelse med en fusion, er det forventeligt at DTUs økonomi vil udhules gennem overførsler til fusionspartnerne. Det vil føre til en dårligere ingeniøruddannelse, hvis der ikke længere kan investeres i de nødvendige faciliteter, såsom laboratorieudstyr og teknik, samt bibeholdes høj konfrontationstid. Netop det faktum, at uddannelserne på DTU i meget høj grad er baseret på eksperimentelt arbejde, gør at vores ingeniøruddannelser er i særklasse.\\

Fusioneres DTU med uddannelsesinstitutioner med lavere STÅ-bevillinger, kan det ikke undgås, at der kommer et massivt økonomisk pres på vores uddannelser. Selvom der kan argumenteres for, at ingeniøruddannelsernes STÅ- midler fortsat udelukkende går til ingeniøruddannelserne, vil det på længere sigt være svært at stå for dette pres. Med en institutstruktur, hvor der på et institut gives undervisning til uddannelser med forskellige STÅ-bevillinger, måske endda fælleskurser udbudt til studerende på forskellige uddannelser , vil det være umuligt at opretholde en fordelingsnøgle, som tager hensyn til de forskellige bevillinger. Der bør findes en holdbar løsning på dette problem, før en fusion igangsættes.

\subsubsection{Fysisk sammenlægning}
PF ser en fysisk sammenlægning af fusionsparterne på et samlet campus som en nødvendighed for, at både forskningen og uddannelserne får bedre vilkår. Fysisk sammenlægning af universiteter og sektorforskningsinstitutioner vil betyde en større kvalitet i forskning og vil kunne sikre kritisk masse for det samlede universitet i form af højt antal aktive forskere og deres konkurrencedygtighed. Hvor det er relevant, er der allerede nu samarbejde mellem både små og store forskningsmiljøer, og en fysisk sammenlægning mellem de rette institutioner vil kunne give et nyt brydningsfelt og åbne op for en række muligheder for fremtiden. Vidensdelingen, konkurrencen og tværfagligheden skal styrkes i innovationens navn.\\

For uddannelserne vil en fusion kun hæve kvaliteten, hvis der sker fysisk sammenlægning. Kursusudbuddet for den enkelte studerende skal øges, og mulighederne for tværfaglighed styrkes. Dette indebærer naturligvis ensretning af semester- og skemastrukturer. En eventuel ændring af semester- og skemastruktur må ikke gå på kompromis med de nuværende elementer i DTUs uddannelser og må kun gennemføres, hvis det fører til en forbedring af uddannelsen. PF ønsker at tage del i alle diskussioner om DTUs uddannelser, og forventer indflydelse på eventuelle ændringer.\\

Mange uddannelser vil komme til at lide under en fusion uden entydig fysisk sammenlægning, da en samling af enkelte fælles forskningsmiljøer vil flytte fagområder væk fra mange studerende. Således bliver studerende fysisk adskilt fra forskningsmiljøerne og vil miste det forskningsnære undervisningsmiljø, hvilket endvidere vil kunne føre til at studerende fravælger disse fagområder og kursustyper. Dette kan betyde en mindre tværfaglighed og en sænkning af kvaliteten.\\

Bliver fusionen ikke fysisk, vil studerende skulle pendle mellem forskellige campus-afdelinger for at modtage undervisning. Der er fare for, at de studerende får ringere tilhørsforhold til deres uddannelsesinstitution, hvilket ikke er hensigtsmæssigt. Skulle studerende som del af deres uddannelse pendle mellem forskellige afdelinger skal det foregå med direkte og gratis transport.\\

At lade underviserne pendle er en bedre mulighed, men en manglende kontakt med underviserne i dagligdagen vil blive et problem for mange projektskrivende studerende. Desuden er det en meget dårlig udnyttelse af ressourcerne, hvis undervisere skal bruge arbejdstiden på at rejse frem og tilbage mellem uddannelsesinstitutionerne.

\subsubsection{Studiemiljø på Campus}
De studerendes studie- og arbejdsmiljømæssige forhold må ikke forringes ved en fusion.
PF mener, at
\begin{itemize}
\item[-] De studerende skal have den samme adgang til IT- og databarer som i dag.
\item[-] Alle studerende skal at få stillet et kontor til rådighed i forbindelse med eksamensprojektet.
\item[-] Al undervisning skal foregå under gode forhold i både auditorier og grupperum.
\item[-] Der skal være et eller flere fælles sociale og faglige mødesteder, hvor alle studerende kan få plads.
\end{itemize}

For at skabe et aktivt og levende campus skal forskellige grupper af studerende integreres. Der skal foretages tiltag både i og uden for undervisningen, hvis dette skal lykkes, men den faglige relevans kommer naturligvis før social integration. PF vil desuden gennem bredt studentersocialt arbejde bidrage til processen.\\

I forbindelse med fysisk sammenlægning er det sandsynligt, at presset på kollegierne øges. PF vil samarbejde med DTU, kommunen og erhvervslivet om opførsel af nye kollegier. PF vil endvidere arbejde for, at alle studerende på Sletten kan få stillet værelse til rådighed på PF-kollegierne, såfremt de ønsker det.\\

Det er ikke acceptabelt at studerendes leveomkostninger skal forøges, hverken i forbindelse med transport eller bolig i
forbindelse med en fusion.