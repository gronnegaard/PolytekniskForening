\section{Politikpapir omkring eksamensevaluering}

\subsection{Eksamensevaluering på DTU}
Dette papir definerer Polyteknisk Forenings holdning til eksamensevaluering på DTU.
På nuværende tidspunkt, år 2011, er det ikke muligt for studerende at evaluere deres eksaminer, selvom eksamensevaluering er den bedste mulighed for at give undervisere feedback på eksamen samt give studienævn mulighed for at evaluere kurser i deres helhed. Eksamensevaluering vil også kunne vise om eksamensformen og -niveauet er hensigtmæssige i forhold til at vurdere om læringsmålene i kurset er blevet opfyldt.

\textsl{Polyteknisk Forening ønsker eksamensevaluering på alle kurser.}

\subsection{Eksamensevalueringens indhold}
Eksamensevaluering bør være et redskab til forbedring af undervisningen og bør derfor have et format som understøtter dette. Skemaet bør være tilgængeligt umiddelbart efter eksamenen og antallet af spørgsmål bør holdes på et minimum så flest mulige studerende besvarer alle spørgsmål. Der skal være en rubrik til kvalitative kommentarer, da erfaring har vist at det er denne del af evalueringer som er mest brugbare til udarbejdning af forbedringer. Yderligere bør der i skemaet henvises til læringsmålene for det enkelte kurser så alle studerende har fuldt kendskab til disse.

\textsl{Polyteknisk Forening ønsker at evalueringen sikrer en sammenligning af de givne læringsmål og det eksaminerede.}
\\
\\
\subsection{Mundtlig og skriftlig eksamen}
Der bør ikke være nogen forskel i den principielle udformning af evalueringen for mundtlige og skriftlige eksaminer idet begge eksamensformer er lige gyldige. Ved mundtlige eksaminer kan evalueringen bære præg af den givne karakter, men det forventes at studienævnene er kompetente i deres arbejde og forstår at uddrage det relevante.
\\
\\
\textsl{Polyteknisk Forening ønsker at alle evalueringer bliver behandlet ligeligt.}
\\
\\
\textit{Godkendt af UddannelsesPolitisk Råd 14. februar 2012}