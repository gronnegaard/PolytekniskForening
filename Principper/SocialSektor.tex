\section{Principper for Socialsektoren i Polyteknisk Forening}
I dette dokument præsenteres Polyteknisk Forenings holdning til emner indenfor studiemiljø, bolig og SU. Principperne er udarbejdet og godkendt at Socialudvalget, der er ansvarlig for Polyteknisk Forenings arbejde indenfor disse områder.
\subsection{Principper for studiemiljøet på DTU}
\subsubsection{Sociale aktiviteter}
\begin{itemize}
\item Gennem sociale tilbud skal PF give de studerende mulighed for at være socialt engagerede på DTU uden for undervisningstiden.
\item PF tilstræber årligt at afholde mindst et sportsarrangement, samt en sensommerfest for de studerende.
\item PF arbejder for at alle nystartende skal tilbydes et studiestartsforløb der introducerer til både det faglige og sociale miljø på DTU.
\item PF arbejder for at forbedre de rekreative områder på DTU og fuld tilgængelighed af dem for de studerende
\end{itemize}

\subsubsection{Undervisningsmiljø}
\begin{itemize}
\item Bygning 101 skal være et studentercentrum.
\item Alle områder på DTU skal virke åbne og imødekommende.
\item Alle studerende skal have adgang til gruppelokaler i forbindelse med undervisningen og projekter.
\item  Der skal være læsesale og stillearbejdspladser tilgængelige for de studerende og adgang til disse hele døgnet.
\item Alle eksamensprojektstuderende skal have eget skrivebord og adgang til dette hele døgnet.
\item Underviserne skal respektere de studerende og omvendt.
\item Der skal være aflåselige skabe tilgængelige for de studerende.
\item De studerende skal have mulighed for en frokost af en god kvalitet og til en fornuftig pris.
\item De studerende skal have mulighed for at købe forfriskninger på DTU, for eksempel i automater.
\end{itemize}

\subsubsection{Arbejdsmiljø}
\begin{itemize}
\item Arbejdsmiljøet på DTU må ikke forringes, men skal kontinuerligt forbedres.
\item Det ergonomiske arbejdsmiljø på DTU må ikke forårsage arbejdsskader på de studerende.
\item Indeklimaet og belysningen på DTU skal leve op til gældende At-vejledninger fra Arbejdstilsynet for indeklima og belysning i undervisningsmiljøer.
\item Arbejdsbelastningen for DTU studerende kan ikke forventes at overstige 30 ECTS point pr. semester.
\item Det skal tilstræbes, at de studerendes arbejdsmiljø overholder arbejdsmiljøloven og som minimum overholder undervisningsloven.
\item Studerende, der arbejder på laboratorium eller værksted, skal have et sikkerhedskursus.
\end{itemize}

\subsubsection{Brug af IT}
\begin{itemize}
\item Alle studerende skal have adgang til en computer eller arbejdsstation for bærbare PC’er, når de skriver opgave, rapport eller lignende.
\item Der skal være fuld dækning af trådløst netværk tilgængeligt for alle studerende på hele DTU.
\item Der skal være net- og strømstik tilgængeligt i alle arbejdsområder på DTU, til studerende med en bærbar computer.
\item Software, der skal anvendes i forbindelse med undervisningen, skal stilles til rådighed for de studerende i hele døgnet.
\item Der skal tilstræbes at alle auditorier har bordplads til anvendelse af en bærbar i forbindelse med en forelæsning.
\end{itemize}

\subsubsection{Internationale studerende}
\begin{itemize}
\item Vi skal behandle de internationale studerende, som vi selv vil behandles i udlandet.
\item De internationale studerende skal tilbydes en Buddyordning.
\item Internationale studerende er velkomne på DTU, men må ikke blive en så dominerende gruppe, at danske studerende nedprioriteres.
\item Det tilstræbes at de internationale studerende har adgang til vejledning om studie, bolig og arbejdsforhold på lige plan med de danske studerende.
\item Det skal tilstræbes at internationale studerende får bedre information om arrangementer på DTU.
\end{itemize}

\subsection{Boligprincipper for Polyteknisk Forening}
\subsubsection{PF arbejder for, at:}
\begin{itemize}
\item Der er boliger af en vis standard til alle studerende ved DTU
\item Studerende kan få bolig tæt ved DTU
\item Udgifter til en studiebolig skal stå i rimeligt forhold til en studerendes indkomst
\item Der indstilles til boligerne efter behovskriterier, herunder
	\begin{itemize}
	\item Transporttid til uddannelsesinstitutionen
	\item Sociale forhold
	\item Økonomiske forhold
	\end{itemize}
\item Der er lige indstilling for alle studerende, uanset studie og køn
\item Der ved studieboligen gives mulighed for et godt socialt miljø
\item Nybyggerier skal være fremtidssikrede bl.a. mht.:
	\begin{itemize}
\item  Fællesfaciliteter
\item Drift og vedligehold
\item Tekniske forbedringer
	\end{itemize}
\item At maksimalt $10\%$ af beboerne på et kollegium er internationale studerende
\end{itemize}

\subsubsection{Indenfor boliger til internationale studerende arbejder PF for:}
\begin{itemize}
\item At DTU kan tilbyde boliger til internationale studerende under hele deres studieophold på DTU
\item At der skal opføres et permanent internationalt kollegium på DTU for at sikre de internationale studerende bedst mulige bolig- og sociale forhold tæt på studiet.
\item At det midlertidige kollegium Campus Village nedlægges i forbindelse med indvielse af et permanent internationalt kollegium.
\item At master studerende har fortrinsret til værelser fra det internationale kontor på almindelige kollegier.
\item At det bliver muligt for både danske og internationale studerende at ansøge om, at bo på et internationalt kollegium.
\end{itemize}

\subsection{Principper for Buddyordningen}
DTU finansierer en buddyordning for internationale studerende. PF administrerer ordningen og står for ansættelse og lønning af et antal buddykoordinatorer. Formålet med Buddyordningen er at sikre at der med jævne mellemrum bliver afholdt arrangementer for de internationale studerende på DTU. Buddykoordinatorerne er ansvarlige for at planlægge og afholde disse arrangementer, herunder også at reklamere for dem.\\

Det er ambitionen at der skal afholdes minimum 3 arrangementer hvert semester. Arrangementerne skal hovedsageligt være placeret i 13-ugers perioden og gerne i starten af denne. Eksempler på arrangementer kan være studieture, kulturelle aftener eller udflugter, tema fester og sociale arrangementer med danske studerende.\\

Buddykoordinatorerne er desuden faste medlemmer af PFs internationale udvalg. Medlemskab af dette udvalg skal dels sikre at Buddykoordinatorerne ved hvilke initiativer der arbejdes med for de internationale studerende og dels sikre udvalget et naturligt rekrutteringsgrundlag for nye medlemmer gennem Buddykoordinatorernes gode kontakt til de internationale studerende.

\subsection{SU-principper for Polyteknisk Forening}
Statens Uddannelsesstøtte (SU) skal sikre, at alle danskere har mulighed for at uddanne sig med støtte fra samfundet. Denne gode er nødvendig, for at mange studerende påbegynder en videregående uddannelse. Det er derfor helt centralt, at holde fast i SU’en, for at sikre fortsat fornuftige levevilkår for studerende ved DTU.
\begin{itemize}
\item PF ønsker en SU, der er høj nok til at de studerende kan klare sig udelukkende af denne.
\item PF arbejder for fortsat at bevare ekstra støtte til studerende med børn og dårligt stillede studerende.
\item Indenfor SU-lån er der i dag kun mulighed for at låne et fast beløb pr. måned eller ingenting. PF ønsker en større fleksibilitet i SU-systemet så det bliver muligt at vælge mellem flere låne størrelser.
\item PF mener ikke at det er hensigtsmæssigt, at anvende SU’en som et incitamentsstyrende redskab. For at sikre, at alle kan modtage en uddannelse ved DTU under økonomiske forhold, der kan hænge sammen, mener PF derfor at det er nødvendigt at vedholde en fast økonomisk SU-sats for alle.
\item PF mener at det er nødvendigt, at SU’en fortsat er tilpasset studiets normerede længde, så der er mulighed for at tage alle uddannelser, med en vis margin. Derudover er det vigtigt at holde fast ved, at de studerende har muligheden for at melde SU’en fra i perioder, hvor der er mere plads i budgettet, så der ikke er nogle, som i sidste ende står uden SU under eksamensprojektet.
\end{itemize}

\subsection{Studiestart for Ingeniørstuderende ved DTU}
(Vejleder og Vektor er i det følgende formuleret som Rusvejleder)\\
Dette dokument indeholder ønsker og krav til Polyteknisk Forenings studiestartsaktiviteter. Dokumentet er udarbejdet som et arbejdsdokument for FR, PF’s bestyrelse og de studerende i den koordinerende studiestartsgruppe, således at der er klarhed over ansvar og arbejdsfordeling.\\

Polyteknisk Forenings studiestartsaktiviteter varetages af en koordinerende gruppe, der dækker både Diplom- og Civilingeniøruddannelserne. Den koordinerende gruppe er ansvarlig for optag af diplomingeniørstuderende i både februar og september, samt optaget af civilingeniørstuderende til grunduddannelsen i september.

\subsubsection{Den koordinerende gruppe}
Den koordinerende gruppe nedsættes som aktivitetsudvalg under Fællesrådet som Koordinering Af Bachelor Studiestart (KABS) efter individuelt valg af de enkelte studerende i faglige råd og fællesrådet. Konstitueringen af dette udvalg sker på baggrund af et fælles fokuspapir, hvor der redegøres for den koordinerende gruppes ideer for afvikling af rusture, forløbet efter rusturen og uddannelse af rusvejledere, samt retningslinjer for studiestarten. Aktivitetsudvalget refererer herefter til Fællesrådet og holder tæt kontakt til Polyteknisk Forenings bestyrelse. Aktivitetsudvalget står for udvælgelsen og opkvalificeringen af rusvejledere, afvikling af rusture, koordinering af forløbet efter rusturene, samt indgår i samarbejde med DTU omkring tutorordningen. Det er af særligt vigtig karakter at aktivitetsudvalget arbejder langsigtet i forhold til studiestarten som en helhed. Rollen for det koordinerende udvalg er overordnet og et medlem af aktivitetsudvalget kan ikke samtidig fungere som rusvejleder i pågældende studiestart. Arbejdet i aktivitetsudvalget er lønnet. Efter endt tjeneste aflønnes i henhold til ansættelseskontrakten.\\

Det er ønsket at medlemmer af det koordinerende aktivitetsudvalg har erfaring med arbejdet i foreningen, ligesom det forventes at medlemmerne har en grundlæggende forståelse for rusvejlederhvervet.

\subsubsection{Rusvejlederne}
En rusvejleders opgave er at være med til at give de nye studerende den bedst mulige start på studiet, samt arbejde for at støtte og integrere dem socialt og fagligt. Hver rusvejleder vælges til at varetage en gruppe nystartende fra den retning de repræsenterer på rusturen, samt følge denne gruppe gennem det første semester (inkl.3-ugers periode). Rusvejlederne forventes desuden at stå til rådighed for deres russer i løbet af russernes andet semester. På selve rusturen er der mulighed for at opdele de studerende fra en retning, således at der opnås tværfaglige netværk mellem de studerende på en rustur. Et hold af nystartende forventes i størrelsesordenen 8-10 studerende. Rusvejlederhvervet er lønnet, og efter udvælgelsen skal rusvejlederne præsenteres for en ansættelseskontrakt. Efter endt tjeneste aflønnes i henhold til ansættelseskontrakten.

\subsubsection{Uddannelse}
Alle rusvejledere gennemgår inden rustur opkvalificerende kurser. Det er et krav, at alle rusvejledere er uddannet i førstehjælp. Desuden stilles der krav til at rusvejledere uddannes på nogle af bestyrelsen og aktivitetsudvalget fastsatte kurser. Uddannelse af rusvejledere kan foregå både på og udenfor DTU. Det er ønsket at uddannelse i rustur og sociale arrangementer foregår på uddannelsesturen OPtur, mens uddannelse i førstehjælp afholdes på DTU. Den faglige specialisering (kan for eksempel indeholde: taleteknik, gruppedynamik, psykologi, konflikthåndtering) kan afholdes både på DTU og på uddannelsesturen, men tænkes undervist på OPtur. I forbindelse med den årlige uddannelsestur i påsken er det et krav, at både rusvejledere og den koordinerende gruppe deltager. KABS har forpligtigelse og hovedansvar for at arrangere turen.\\

Det er formålet med uddannelsesturen at give rusvejlederne et grundigt kendskab til, hvad der kræves for at blive en god rusvejleder. Uddannelsesturen OPtur spænder over 4 dage. Uddannelsesturen i forbindelse med vinteroptag på diplomuddannelserne arrangeres af KABS. Rusvejlederne fra delte uddannelser (f.eks. Medicin og Teknologi) kan, hvis de ønsker, deltage på uddannelsesturen, eventuelt mod forhøjet gebyr. Dette forhøjede gebyr fastsættes i samarbejde med uddannelsesinstitutionerne.

\subsubsection{Rusturene}
Det er aktivitetsudvalgets ansvar, at der bliver afholdt et antal rusture svarende til antallet af nystartende på DTU. Rusturene tiltænkes en størrelsesorden på 60 til 100 deltagere og har en længde af 3 til 4 overnatninger. Der skal tillige afholdes en weekendrustur som et tilbud til både civil- og diplomingeniørstuderende i weekenden inden sommerrusturene. Et medlem af aktivitetsudvalget indgår på et hold af rusvejledere i planlægning af en rustur, samt en forsvarlig afvikling af denne. Tillige er det aktivitetsudvalgets ansvar, at de nystartende bliver vist rundt på DTU, får relevant information på rusturene, samt at rusturene har en faglig karakter. Dette gøres ved at invitere forskellige eksterne foredragsholdere (PF’s bestyrelse, IDA, Rektor/Direktør/Dekan, Studenterpræst, Studenterrådgivning, Studievejledning, osv.) til at holde oplæg i løbet af rustursugen (eventuelt på DTU). Det tilsigtes at besøg ikke har karakter af reklamefremstød for kommercielle produkter mod de studerende. Det er ligeledes aktivitetsudvalgets opgave at sørge for at PF’s regler om hytteture, samt dansk lovgivning overholdes på turene. Dette gælder regler bl.a. omkring nøgenhed, ædruansvarlig, stødende/voldelig adfærd og indtagelse af ulovlige stoffer. Alle rusturshold uarbejder i den forbindelse et etisk og moralsk regelsæt, hvori de redegør for deres ideer for en ansvarlig afvikling af rusturen. Disse supplerende regelsæt skal godkendes af de koordinerende udvalg.\\

Det er vigtigt at alle Polyteknisk Forenings rusvejledere har en forståelse for, at rusturen er for de nystartende. Polyteknisk Forening ønsker at der er plads til alle nye studerende på DTU’s rusture, herunder studerende der er interesserede i rolige sociale rammer i forbindelse med opstart af et studie. Det er rusvejlederne og KABS’s ansvar at disse studerende har et
alternativ til de arrangementer der involverer alkohol i løbet af hele studiestartsperioden. Det er ønsket at der ikke er forskel i den måde hvor rusvejlederne på civil- og diplomuddannelserne afholder rusturene. Det er tanken at begge uddannelsers studiestart blandes således at der opnås et tværfagligt netværk i mellem de enkelte retninger på uddannelsen. Dette gøres for at sikre at alle rusture har nogenlunde ensartet karakter af hensyn til de nystartende, men ligeledes af hensyn til rusvejledere. Aktivitetsudvalget kan vælge at bede et passende antal (2-3 pr. 60 studerende) studerende om at assistere en rustur i forbindelse med madlavning, såfremt at det vurderes at dette samlet set kan forbedre kvaliteten af den enkelte rustur. De 2 studerende er ikke lønnede, men skal følge det visionsoplæg som gælder for rusturen. Aktivitetsudvalget har til opgave at sætte nogle rammer, således at køkkenholdet på forhånd har en forståelse af hvad der forventes af dem.

\subsubsection{Forløbet efter rusturene}
De nystartende, der ikke har deltaget på rusturene, skal efterfølgende kunne indgå i rusvejlederens gruppe som ligestillede med dem, der har været af sted. Efter rustur afholdes møder med russerne, der har til formål at skabe sammenhold og trygge sociale rammer. Desuden kan rusvejlederen formidle relevant information om studiet og besvare eventuelle spørgsmål. Det er et krav, at rusvejlederne afholder ugentlige møder og arrangerer minimum to sociale arrangementer for de nystartende inden for denne periode. Rusvejlederne forventes desuden at tilbyde deres russer to møder i løbet af
russernes andet semester.\\

Desuden stilles der krav til afholdelse af en fagaften for retningen, hvor rusvejledere og undervisere præsenterer de
muligheder der er for kurser og studieforløb på de efterfølgende semestre. Det er aktivitetsudvalgets ansvar at kontrollere om de faglige og sociale arrangementer udføres tilfredsstillende. Desuden har rusvejlederen et ansvar for at afholde sociale arrangementer for alle de nystartende som denne har været på rustur med, således at de nystartendes tværfaglige netværk fra rusturen vedligeholdes. Aktivitetsudvalget står desuden for kontakten til involverede institutter og studieledere, og indgår i et samarbejde med disse om at udvælge passende tutorer.

\subsubsection{Evaluering og overlevering}
Aktivitetsudvalget har til opgave at lave en evaluering af rusturene og det samlede rusvejlederforløb blandt de nystartende. Desuden udarbejdes evalueringer af rusvejledernes personlige præstationer af aktivitetsudvalget. Evalueringerne behandles i aktivitetsudvalget, samt samles i en detaljeret overlevering til det efterfølgende udvalg. Tilbagemelding til FR omkring resultaterne i forbindelse med evaluering af uddannelsesturen, rusture og efterfølgende forløb sker skriftligt. Et Emne Udvalg Evaluerings Udvalg (KABSeueu) nedsættes af fællesrådet i oktober i forbindelse med valg af nye koordinatorer. Dette udvalg indeholder medlemmer af FR og det allerede siddende aktivitetsudvalg. I forbindelse med valg af KABSeueu fastsætter FR visioner for det kommende års studiestartsaktiviteter