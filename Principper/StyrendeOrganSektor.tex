\section{Styrende organ sektoren}

\subsection{Politikpapir om forholdet imellem Diplom- og Civilingeniøruddannelserne}
Baggrund for diskussionen
Diplom- og civilingeniøruddannelserne har tidligere været udbudt på to forskellige uddannelses-institutioner, men dette
blev ændret i 1995, hvor DTH, DIA og Helsingør Teknikum, IHE blev lagt sammen til Danmarks Tekniske Universitet.
Det var med til at starte diskussionen af samarbejdet og forholdet imellem diplom- og civilingeniøruddannelserne.
I begyndelsen beholdt diplomingeniør-uddannelserne de lokaler, der før blev benyttet af DIA, og der var derved en
fysisk adskillelse af de to uddannelser. Denne adskillelse er forsvundet i løbet af de sidste par år, hvor DTU’s ledelse
har gennemført en fortætning af DTU’s institutter. Herved er diplomuddannelserne flyttet til det/de institut(ter), de har
en faglig relation til. Der er herved lagt op til et endnu tættere samarbejde imellem de to uddannelser. Nogle institutter
har for eksempel oprettet fælleskurser imellem diplom- og civilingeniøruddannelserne. Flere af de såkaldte
diplomundervisere er samtidig ved at gå pension, så man ikke længere kan tale om en adskillelse imellem diplom- og
civil undervisere.
Fra sommeroptaget 2004 er civilingeniøruddannelsen delt op i en 3-årig bachelor i teknisk videnskab og en 2-årig
kandidatuddannelse for herved at følge Bologna-deklarationen og uddannelsesbekendtgørelsen. Dette har sat yderligere
skub i diskussion om, hvorvidt DTU skal udbyde to uddannelser, og hvad forskellene på disse er.
To uddannelser på DTU
PF mener, at DTU skal udbyde såvel diplom- og civilingeniøruddannelserne, så længe disse er forskellige. Denne
politik er således kun gældende, hvis bachelordelen af civilingeniøruddannelsen ikke nødvendigvis skal give
erhvervskompetence, hvorimod dette skal være gældende for diplomingeniøruddannelsen.
En meget vigtig begrundelse for at opretholde de to uddannelser er, at de appellerer til forskellige typer af studerende.
Der er forskellige indlæringsformer og studieopbygning, som også er forklaret under afsnittene for hver uddannelse. De
studerende, der starter på de to uddannelser, har således forskellige forventninger, krav og indgangsvinkel til
uddannelsesforløbet. PF mener, at de studerende fortsat skal have muligheden for at vælge den uddannelse, der passer
den enkelte studerende bedst, og derfor skal DTU fortsat udbyde to uddannelser. Derudover kommer forskellen i de
færdiguddannede ingeniørers kompetencer, der retter sig mod forskellige dele af erhvervslivets behov.
PF mener, at det er vigtigt, at fleksibiliteten imellem uddannelserne på DTU opretholdes. PF vil derfor arbejde for, at
det er muligt at foretage meritoverførsel imellem de to uddannelser, hvor dette er fagligt relevant.
PF mener ikke, at der skal være fælleskurser imellem diplom- og civilingeniøruddannelserne ved obligatoriske dele af
uddannelserne.
PF vil arbejde for at såvel studerende som ansatte ved DTU betragter diplom- og civilingeniøruddannelserne som
ligeværdige.
Diplomingeniøruddannelsen
Diplomingeniøruddannelsen skal være en anvendelsesorienteret uddannelse, der giver umiddelbar erhvervskompetence
efter endt studie. Studieplanen for diplomingeniørstuderende skal være fastlagt for størstedelen af uddannelsen. PF
mener, det er vigtigt, at den studerende kan gennemføre studiet på normeret tid med 30 ECTS point pr. semester ved at
følge studieplanen og de anbefalede studieforløb.
Der skal være klasseundervisning på minimum det første år af diplomingeniøruddannelsen. For at der er tale om
klasseundervisning, må klasserne ikke overskride 40 studerende. Undervisningen skal være anvendelsesorienteret, og
der skal lægges vægt på det praktiske indhold af fagområdet i såvel undervisning som projektarbejde.
For at sikre den anvendelsesorienterede profil på diplomingeniøruddannelsen mener PF, at en vis del af underviserne
bør have erhvervserfaring eller være gæsteforelæsere fra erhvervslivet. Det er dog vigtigt at undervisningen ikke retter
den studerende mod en enkelt virksomhed. Diplomingeniøruddannelsen skal indeholde en obligatorisk praktikperiode,
der er fagligt relevant.
Bacheloruddannelsen i teknisk videnskab
Bacheloruddannelsen skal forberede den studerende på at læse videre på kandidatuddannelsen ved at give den
studerende en teoretisk-naturvidenskabelig grundforståelse. PF støtter ikke, at man tager kurser med
erhvervskompetence på bacheloruddannelsen, da primær fokus skal ligge på, at man skal fortsætte på
kandidatuddannelsen.

PF vil arbejde for at minimum $\frac{1}{4}$ af uddannelsen skal bestå af valgfrie kurser. Med valgfrie kurser menes her total
valgfrihed, hvor man kan vælge frit imellem hele kursusudbudet på DTU (eller andre relevante universiteter), og ikke
blot valgfrihed indenfor rammerne af et enkelt kursus eller en gruppe af kurser.
Undervisningen på bacheloruddannelsen skal være forskningsbaseret, og der skal lægges vægt på det teoretiske indhold
i såvel undervisningen som ved projekter. Praktikophold skal ikke være obligatorisk på bacheloruddannelsen, men den
studerende skal have mulighed for at tage i praktik og få dette meritoverført.
Godkendt af UDU d. 17/1 2005
FR orienteret på FR120 d. 20/3 2005
\subsection{Sprogpolitik på DTU}
Internationalisering er i stigende grad ved at blive en del af hverdagen på DTU og i samfundet generelt. Der er både
positive og negative effekter af denne internationalisering, og dette oplæg vil diskutere problematikken omkring
engelsksproget undervisning på DTU.
Mængden af kurser, der på DTU udbydes på engelsk, stiger, og der er fra flere sider ønske om, at denne tendens skal
fortsætte. I PF har problematikken været diskuteret i UddannelsesUdvalget (UDU) 17. januar 2005 og dette oplæg
bygger på holdningerne, der fremkom på mødet.
Det er vigtigt, at man i løbet af studiet opnår et dansk fagsprog, således at man kan indgå optimalt på dansksprogede
arbejdspladser. Desuden er det danske sprog vigtigt for, at man kan udtrykke sig og formidle information til folk fra
andre fag eller til befolkningen generelt. Vi mener derfor, at det er hensigtsmæssigt, at den obligatoriske del af
bacheloruddannelsen og diplomingeniøruddannelsen foregår på dansk. Det vil betyde, at man som studerende opnår et
bredt fundament med danske fagtermer og evnen til at formulere sig på dansk. Da indlæringen desuden er højest på
modersmålet, er det også vigtigt, at sproget er dansk i starten af studiet, hvor man i forvejen skal vænne sig til et nyt
læringsmiljø og undervisningsform.
På kandidatuddannelsen derimod vil det være hensigtsmæssigt, at der indgår kurser på engelsk, således at man også
opbygger et engelsk fagsprog. I mange virksomheder er hovedsproget engelsk, og det er derfor vigtigt at kunne
formulere sig og deltage i faglige diskussioner på engelsk. Desuden skal det være muligt at rejse til udlandet og arbejde,
hvorfor et basalt engelskkendskab er vigtigt.
PF mener at kandidatuddannelser på engelsk vil være fornuftigt. Kurser, der udbydes på engelsk, skal dog kun afholdes
på engelsk, hvis der er internationale studerende til stede i undervisningen, eller hvis underviseren ikke er dansktalende.
Hvis kun danske studerende er til stede, bør kurset afholdes på dansk for at sikre det højst mulige udbytte af
undervisningen.
Fordelen er, at de danske studerendes engelskkundskaber forbedres, at flere udenlandske studerende vil/kan studere på
DTU og international anerkendelse i højere grad vil kunne opnås. Der er dog en række punkter, hvorpå engelsksproget
undervisning er problematisk. Problematikkerne nævnes i dette oplæg, da vi mener, det er vigtigt at have fokus på og
forsøge at afhjælpe/forbedre disse, hvor det er muligt. Det er problematisk med engelsksproget undervisning, da det
stiller krav til, at de studerende har relativt gode engelskkundskaber og ikke kun en teknisk viden inden optagelse på
studiet. Det kan dermed afholde nogle ellers dygtige studerende fra at søge ind på DTU. Læringsudbyttet af engelske
kurser vil være lavere end på danske kurser, da noget information vil gå tabt og kommunikationen besværes. Et andet
problem er, at nogle underviseres engelskniveau er så ringe, at det ikke kan forsvares, at de underviser på engelsk.
Engelsksprogede kurser kræver derfor at undervisere, hvis engelskkundskaber er for lave, opkvalificeres først.
Afslutningsvis er der en anden problematik, som ikke direkte udspringer af den engelsksprogede undervisning, men
opstår i engelsksprogede kurser: Studerende fra udlandet har ofte et meget lavt engelskniveau. Dette er problematisk i
forbindelse med gruppearbejde og rapportskrivning, studenterfremlæggelser og evt. i forelæsningssituationer. Der bør
derfor stilles høje krav til udenlandske studerendes engelskkundskaber og evt. udbydes intensive engelskkurser før
semesterstart.
Det er blevet diskuteret at omdanne enkelte bacheloruddannelser til rent engelsksproget bachelorlinier. Dette bør
overvejes nøje, da det kan have stor betydning for de danske kandidater. F.eks. er en stor del af aftagerne til
miljøingeniører kommuner og amter, hvor skriftlig og mundtlig kommunikation i overvejende grad er på dansk.
Omvendt kan det være en fordel at skabe en rent engelsksproget bacheloruddannelse, da det vil skabe et internationalt
studiemiljø og uddanne ingeniører til i høj grad at kunne indgå i det internationale arbejdsmarked. Vi mener, at DTU’s
vigtigste rolle er at uddanne kandidater til det danske arbejdsmarked, og deres behov bør derfor afsøges, før man ændrer
20
en uddannelse fuldstændigt fra dansk til engelsk. Afslutningsvis er der tre områder, der ikke direkte berører den
engelsksprogede undervisning, men som har med internationaliseringen at gøre, som vi mener at der bør være fokus på.
For kurser hvor undervisningssproget ikke svarer til sproget benyttet i kursusmaterialet, skal der foreligge en
oversættelse af de centrale begreber. Dette skal afhjælpe begrebsforvirring og give mere helhed i læringsprocessen. For
kurser afholdt på engelsk skal det være et krav, at ALT undervisningsmateriale forefindes på engelsk. Institutterne kan
beslutte at anvende materiale på andre sprog end dansk og engelsk, såfremt dette anvendes indenfor det pågældende
fagområde. Sproget for såvel undervisning som kursusmateriale skal fremgå af kursusbeskrivelserne.
Det skal være nemt at tage dele af sin uddannelse i udlandet, eftersom udlandsophold er den mest effektive måde at
opkvalificere sine sprogkundskaber på. Dette indebærer, at studiestrukturen skal være så elastisk og fleksibel, at det
ikke stiller hindringer i vejen for udlandsophold, at der skal arbejdes for at bibeholde og fremme internationale
samarbejdsaftaler og at meritoverførsler skal være gennemskuelige og brugbare på et eksamenspapir fra DTU (f.eks.
bør kursusnavne angives).
Det skal være muligt at aflevere opgaver og eksamenssæt på både dansk og engelsk uanset undervisningssproget, hvis
underviseren behersker begge sprog. Dermed kan studerende, der føler et behov for at arbejde med deres sproglige
færdigheder på enten dansk eller engelsk gøre dette underordnet sproget i det enkelte kursus.
Godkendt af UDU d. 22. Februar 2005
FR orienteret på FR120 d. 20/3 2005
\subsection{Politik for Praktik på diplomingeniøruddannelserne}
Polyteknisk Forening mener at praktikopholdet bør opprioriteres og at der skal skabes fælles retningslinier og øget
kvalitetssikring af praktikopholdet for DTU’s diplomingeniøruddannelser.
Praktikopholdet er en essentiel og karakteriserende del af diplomingeniøruddannelsen, grundet uddannelsens
erhvervsorienterede sigte. Gennem praktikopholdet kan de studerende opnå kompetencer som det ikke er muligt at få
gennem det øvrige studie. Det er derfor vigtigt at sikre en høj kvalitet af praktikforløbet, således at alle studerende får et
optimalt udbytte ud af forløbet. Dette indebærer at der opstilles klare mål med praktikken - for alle involverede parter,
samt at rammerne for praktikforløbet udvikles, så en mere systematisk kvalitetssikring af praktikken kan opnås.
Nedenfor skitseres de områder som PF finder vigtige for at højne kvaliteten af praktikopholdet.
Klar målsætning fra DTU
Det er vigtigt at DTU som uddannelsesinstitution har en klar målsætning for praktikopholdet. Det overordnede formål
og rammer bør være fælles for alle studerende på DTU’s diplomingeniøruddannelser. I udarbejdelsen af målsætningen
er det vigtigt at studerende, undervisere samt repræsentanter fra erhvervslivet inddrages. Målsætningen bør benyttes
som rettesnor i arbejdet med praktikken og bør jævnligt revurderes. Målbeskrivelsen skal indarbejdes til at være en
aktiv del i den daglige planlægning og information vedrørende praktikopholdet.
Bevidsthed blandt de studerende
Det er vigtigt at målene for praktikopholdet er kendte af de studerende. Klare mål skal være med til at sikre, at de
studerende bliver mere bevidstgjorte om praktikopholdet; dets formål og hvilke kompetencer der skal opnås. En øget
bevidsthed blandt de studerende vil lede til et større engagement og dermed et øget udbytte af praktikken.
Kontakt mellem parterne
Kontakten mellem de involverede parter; studerende, uddannelsesinstitution og virksomheder bør øges. Der skal
foreligge klare individuelle forventninger mellem alle parter. Studerende og virksomheder kan lære meget af hinanden
og det forventes at en mere klar kommunikation vil føre til et bedre samarbejde. Der skal under praktikopholdet være
løbende, fastlagt, kontakt mellem de involverede parter. Desuden skal der konsekvent arbejdes for at udvikle aftaler
med virksomheder og finde nye praktikpladser. Udover at skabe fornyelse og erfaringer, der kan være med til at udvikle
praktikken, vil denne kontakt kunne bidrage til den generelle erhvervsorientering og præge uddannelsen mod
arbejdsmarkedets behov. I det lange løb kan det gøre virksomhederne mere klar over hvad de nyuddannede kan tilbyde,
hvilket gør DTU og de studerende mere attraktive.
Evaluering
Efter endt praktikophold skal der foregå en evaluering. Evalueringen bør foregå ensartet for alle studerende på DTU,
uanset studieretning. Evalueringen skal, udover at indgå som vurderingsgrundlag for bestået praktikophold, give
21
mulighed for refleksion af praktikopholdet for den enkelte studerende. Sidstnævnte skal ses som en del af evalueringen.
Det er vigtigt, at der foreligger krav til de studerende og deres rapporter.
I muligt omfang bør virksomhedernes evaluering af praktikanten også indgå som vurderingsgrundlag. Dette skal dog
ikke være et krav men alene ses som vejledende. Dette vil kunne give den studerende mulighed for yderligere refleksion
af deres praktikophold, men det skal også give mulighed for at øge kontakten mellem parterne.
Evalueringen vil samtidigt kunne bistå den/de praktikansvarlige i at lave en vurdering af særligt egnede eller uegnede
praktikvirksomheder, således at virksomheder ikke gentagne gange modtager praktikanter, hvis de ikke er i stand til at
honorere DTUs forventninger og krav til et praktiksted.
Nedenfor beskrives og uddybes konkrete forslag, der kan være med til at forbedre praktikopholdet. Forslagene skal ses
som en inspirationskilde for den videre udvikling og flere af dem er mulige at implementere direkte. En del af tiltagene
indebærer øget økonomisk tilskud. Vi anser disse tiltag som væsentlige for et reelt kvalitetsløft af praktikken og mener
derfor, at det er nødvendigt at tilføre praktikopholdet de nødvendige ressourcer. Praktikken er en karakteriserende del af
uddannelsen og udgår tidsmæssigt 1/7 af denne og det er derfor uforsvarligt at sløse med kvaliteten af praktikopholdet
Konkrete forslag
 Informationsmøder. Udover praktiske informationer vedrørende praktikopholdet, bør mødet indeholde oplæg
fra virksomheder og ældre studerende der fortæller om deres erfaringer og refleksioner. Det er vigtigt at dette
møde lægger op til, at de studerende selv reflekterer over det kommende praktikophold. Refleksionerne bør
udmunde i en skriftlig eller mundtlig redegørelse. Dette kunne gøres i forbindelse med jobbeskrivelsen, se
nedenunder.
\begin{list}{•}
\item Før selve praktikken starter, skal der foreligge en jobbeskrivelse, der er udarbejdet i samarbejde mellem
    virksomheden, praktikvejlederen og praktikanten, gerne ved et fælles møde. Dette vil være med til at sikre, at
      de opgaver der skal udføres er relevante og af høj kvalitet.
      \item I løbet af praktikken skal der være kontakt mellem de involverede parter. Kontakten skal bestå af minimum et
      besøg på virksomheden fra den tilknyttede vejleder. Besøgene kunne være af tutorerne fra det indledende
         studieår.
         \item Der skal sættes tydelige krav til indholdet af rapporten. Der skal for alle studerende udarbejdes fælles
     retningslinier for rapporten. Retningslinier bør som minimum indeholde krav til den faglige beskrivelse,
     logbog over praktikopholdet, afleveringsfrister. Afslutningsvis bør evalueringen bestå af en mundtlig
      fremlæggelse, gerne for hele retningen.
      \item Som en del af evalueringen skal hver studerende skrive en specifik evaluering af deres praktiksted.
   Evalueringen skal beskrive virksomheden og de opgaver praktikanten har haft i løbet af perioden. Alle
   evalueringerne skal derefter samles i en database. Denne kan senere benyttes som inspiration og vejledning i
  valget af praktikplads for de efterfølgende studerende. Dette vil kunne være med til at sikre at dårlige
    virksomheder fravælges.
\end{list}

Der bør nedsættes en koordinerende gruppe, der konkret arbejder med at skabe fælles retningslinier for praktikopholdet.
Gruppen bør samle erfaringer fra de enkelte retninger. Der skal være studerende repræsenteret.
Desuden foreslår vi, at der kigges på lønforholdene for praktikstuderende. Det bør overvejes, om de samme forhold bør
gælde for alle DTU’s studerende, og hvilke retningslinier uddannelsesinstitutionen samlet melder ud til erhvervslivet.
Godkendt af UDU d. ?
FR orienteret på FR130 d. 27/10 2005

Principper for valg af studenterrepræsentant til DTU’s bestyrelse og for dennes arbejde.
DTU’s ledelsesstruktur lægger op til at vi som studerende skal vælge en repræsentant til DTU’s bestyrelse hvert år. Det er klart at PF ønsker en central person i foreningen valgt til at varetage denne post, fordi det er den mest
magtfulde post en studerende kan bestride i DTU regi. Denne person bør arbejde indenfor nogle principper, som fællesrådet i sidste ende vedtager. 
\\
\\
\textit{Dette oplæg er en revidering af et vedtaget oplæg fra 2000 af Kristian Smistrup.}
\\
\\
Hvilke forudsætninger skal man have
Da arbejdet i DTU’s bestyrelse er af overordnet karakter, skal den opstillede kandidat have indgående kendskab til
DTU’s SO-sektor og gerne nogen viden om landspolitik. Man behøver ikke have et indgående kendskab til alle sektorer
i PF, men det er nødvendigt, at man kender så meget til de enkelte sektorer, at man ved, hvem man skal ringe til, og
hvornår man skal gøre det.
\\
\\
Derfor er det nødvendigt, at vores kandidat er en, der har fulgt godt med i foreningen i flere år, og som kender et bredt
udsnit af foreningens holdninger godt. Det er også grunden til, at man i flere år har valgt primært at lade konsistorium
bestå af tidligere bestyrelsesmedlemmer. Det er i den sammenhæng vigtigt, at vedkommende er indstillet på at gå til
både PF's bestyrelses-, fællesråds- og andre relevante møder.
\\
\\
Det skal også være en, der på tæt hold følger med i, hvad der sker i PF. Selve arbejdet i DTU’s bestyrelse består af at gå
til fire møder om året, men dertil kommer så en masse arbejde, som består i at holde sig opdateret på, hvad der foregår i
foreningen og sørge for at have sit bagland med. Det er også en meget vigtig del af arbejdet at melde tilbage til
foreningen, så der er styr på, hvad man render og laver. Det er essentielt at vedkommende holder tæt
kontakt til PF’s bestyrelse, der tegner foreningen politisk. Kort sagt er PF’s kandidats samarbejdsevner meget vigtige,
især da mange sager i DTU’s bestyrelse er omfattet af forskellige grader af fortrolighed.
Naturligvis skal vedkommende være indforstået med, at han eller hun repræsenterer PF’s holdninger og ikke sine egne,
så længe man sidder i bestyrelsen. Man må dog i PF respektere, at man som bestyrelsesmedlem sidder med et juridisk
og personligt ansvar, så der er grænser for, hvor langt man kan gå for PF. Man skal altså kunne følge et mandat.
På den mere personlige side er det nødvendigt, at det er en, der har bevist, at han eller hun har politisk
gennemslagskraft, og at han eller hun kan tåle at man nogle gange får nogle ”tæsk” både i PF og i DTU’s bestyrelse.
Personen skal også være i stand til at formidle et budskab klart og tydeligt og ikke føle den store benovelse over det fine
selskab. Evnen til hurtigt at kunne analysere sig til fremtidige/langsigtede konsekvenser af forskellige forslag er også
væsentlig.
Andre ønsker til et studentermedlem af DTU’s bestyrelse
Ud over de deciderede forudsætninger er der nogle andre kvalifikationer, som det kunne være rart, vis en kandidat til
bestyrelsen besad. Det er godt at vide noget om organisationer generelt ligesom en hvis erfaring med, hvad det vil sige
at sidde i en bestyrelse kunne være praktisk.
Endelig ville det også være meget godt, hvis den pågældende person kendte noget til, hvordan man gjorde andre steder.
Der findes andre universiteter i verden, der er selvejende, og har en bestyrelse. Det kunne være godt, hvis vores
kandidat vidste noget om, hvordan det er gået disse steder. Endelig kunne det jo være virkelig godt, hvis vedkommende
ligefrem havde nogle kontakter sådanne steder.
Kan man sidde i DTU’s bestyrelse samtidig med PF’s bestyrelse
Ud fra en betragtning om at en studerende i DTU’s bestyrelse formentlig ofte i en forhandlingssituation må gå på
kompromis for blot at få en del af PF’s holdning ind i en beslutning, vil det være temmelig uheldigt, hvis hele PF
herefter skal klandres for at resten af PF ikke bakker op om en beslutning. Derudover kan vedkommende komme ind i
inhabilitetsproblemer hvis DTU’s bestyrelse skulle drøfte noget som PF har en stor interesse i, fx. Boghandlen, S-huset
eller lign.
\\
\\
%Det er i øvrigt også et gammelt princip, at PF’s formand eller øvrige bestyrelsesmedlemmer ikke må sidde i konsistorium på DTU. Grunden til dette er, at man, hvis konsistorialerne får sagt noget forkert til et konsistoriemøde, kan man lade formanden gå over til rektor og fortælle, hvad PF egentlig mener.

Derfor må det være reglen, at man ikke sidder i DTU’s og PF’s bestyrelse samtidig.
Økonomien omkring at sidde i DTU’s bestyrelse
Den studerende i DTU’s bestyrelse er i øjeblikket omfattet af en DTU tegnet forsikring til de interne, således at de ikke
går økonomisk konkurs i tilfælde af at DTU går konkurs eller lign. Det er dog uvist om der måske på et senere tidspunkt
skulle tilfalde de interne medlemmer i bestyrelsen en ekstra godtgørelse. I dette tilfælde opfordres der på kraftigste
til at den studerende i bestyrelsen ikke selv tager imod denne betaling, da det nemt kunne drage tvivl om
vedkommendes intentioner med at sidde i bestyrelsen. For eksempel kunne disse ekstra penge gå til studentersociale
formål, måske i form af et tilskud til PF’s studentersociale fond.
\\
\\
Afrunding
Dette er et oplæg, der gerne skulle gøre, at vi diskuterer nogle principper for at bestride og vælge folk til denne meget
vigtige post. Der er måske lagt op til nærmest umulige forventninger til denne person, men da posten er vigtig for PF,
svær at bestride samt meget ensom kræver posten sin mand eller kvinde.
Vedtaget af FR på FR-møde 83 den 10/10 2002
Polyteknisk Forenings forventninger til vores medlemmer af de styrende organer
Dette afsnit omhandler de forventninger, Polyteknisk Forening har til de studerende, der er valgt ind i DTU’s styrende
organer på PF-lister.
De centrale forventninger kan sammenfattes i det følgende:
\begin{itemize}
\item Medlemmerne skal repræsentere alle Polyteknisk Forenings medlemmer og så vidt muligt arbejde
for alle studerende på DTU.
\item Medlemmerne skal holde god kontakt til Polyteknisk Forenings centrale organer og personer.
\item Medlemmerne kan forvente, at Polyteknisk Forenings bestyrelse sørger for, at de modtager relevant
information.
\item Medlemmerne kan forvente, at der bliver afholdt opkvalificerende seminarer og politikskabende
aktiviteter, hvor de forventes at deltage.
\end{itemize}

Generelle forventninger til medlemmerne
Forventningerne skal ikke opfattes at udgøre en checkliste, hvor kandidaterne bliver idømt en straf, hvis de ikke
opfylder dem. På den anden side forventes det, at de opstillede forud for deres opstilling er blevet gjort bekendt med
disse forventninger og har accepteret dem.
Polyteknisk Forening er repræsenteret i mange forskellige organer, og derfor må forventningerne til repræsentanterne på
forskellige niveauer være forskellige. Dog er her først beskrevet en række generelle forventninger, som er fælles for alle
medlemmer.
Medlemmernes opgave i de kollegiale organer er at arbejde for alle studerende på DTU. Det betyder, at der skal holdes
god kontakt til det relevante bagland. Man skal før hvert møde afholde formøde eller på anden måde have drøftet
punkterne på dagsordenen.
Derudover bør man indlede et samarbejde med undervisere (VIP’ere) og det teknisk administrative personale (TAP’ere)
om emner, hvor man kan have fælles interesser.
Derudover skal man møde op i det organ man er valgt ind i. Som minimum sendes der afbud, men der efterstræbes at
kontakte suppleanten, som har tilsvarende forpligtigelser.
Endelig forventes det, at man følger Polyteknisk Forenings politik, formuleret af enten det faglige råd,
Uddannelsespolitisk Råd, Fællesrådet eller i nærværende principkatalog. Hvis man ikke kender politikken, eller hvis
man er uenig i politikken, skal man tage initiativ til, at punktet bliver behandlet i det relevante organ. I sager, hvor PF
ikke har en formuleret politik, må man naturligvis lade sin sunde fornuft råde.
Forventninger til institutstudienævnsmedlemmer
De forventes at møde op til deres faglige råds møder, ligesom hvert studienævn forventes at være repræsenteret med
mindst én studerende til møder i Uddannelsespolitisk Råd.
I institutstudienævn, hvor flere faglige råd har interesse, er det vigtigt, at udmeldingerne koordineres på forhånd.
Koordineringen skal foregå under hensyn til, hvem der bliver mest berørt af det punkt, der skal behandles. Det giver
ikke mening, at to civilstuderende overtrumfer en diplomstuderende i et spørgsmål, der kun berører de
diplomstuderende.
Det er vigtigt, at institutstudienævns medlemmerne holder kontakt til den uddannelsespolitiske koordinator og den
øvrige Polyteknisk Forenings bestyrelse. Desuden skal de holde kontakt til relevante uddannelsesledere samt deres
baggrundsgrupper, såsom retningsfølgegrupper.
Forventninger til institutstudienævnenes næstformænd
Næstformændene forventes at koordinere alle studerende i institutstudienævnene, hvilket indebærer ansvaret for at:
17
-
-
-
studienævnet er repræsenteret til UPR-møder
der holdes formøder eller koordineres før studienævnsmødet
være de studerendes kontakt til institutlederen og resten af instituttet.
Det forventes, at de studerende ud over at deltage til møder indkaldt af institutlederen også laver opsøgende arbejde for
at medvirke i de for studerende interessante diskussioner. Det forventes at næstformændene mødes med institutlederen
minimum 2 gange pr. semester. Desuden er det næstformændenes opgave at videregive kommentarer og forespørgsler
modtaget fra de studerende til institutlederen og omvendt.
Det forventes, at næstformanden gør sig bekendt med instituttets udviklingsmål og virkemidler (UMV) samt
handlingsplan. Det er vigtigt i forbindelse med udarbejdelsen af UMV’en, at de studerende er opsøgende og selv søger
indflydelse. Dette er naturligvis et tillidsspørgsmål, men ved interesseret og overvejet deltagelse i instituttets daglige
gang er det muligt at komme med på råd.
Det er vigtigt, at næstformændene holder kontakt med den uddannelsespolitiske koordinator og den øvrige bestyrelse
for Polyteknisk Forening.
Forventninger til uddannelsesrådsmedlemmer (CUU og DUU)
Der forventes fuld deltagelse til møder i Uddannelsesudvalgene, samt eventuelle nedsatte underudvalg. Derudover
forventes det, at de deltager i Brutto’s arbejde, og mindst ét medlem af hvert udvalg skal møde op til Fællesrådsmøder.
De forventes at holde god kontakt til Polyteknisk Forenings bestyrelse.
Forventninger til medlemmer af Akademisk Råd (AR)
Der forventes fuld deltagelse til møder i Akademisk råd, samt eventuelle nedsatte underudvalg. Derudover forventes
det, at de deltager i Brutto’s arbejde, og mindst ét medlem skal møde op til Fællesrådsmøder.
Desuden forventes det, at de holder god kontakt til de studerende i DTU’s bestyrelse og Polyteknisk Forenings
bestyrelse.
Hvilke krav kan medlemmerne stille til Polyteknisk Forening som organisation
Polyteknisk Forening skal fungere som videns- og informationsbase. Bestyrelsen har ansvaret for, at relevant info
kommer videre til rette vedkommende.
Polyteknisk Forening står bag medlemmerne, hvis der opstår problemerne i forbindelse med at føre PF’s politik.
Det er vigtigt, at der informeres om, hvad Polyteknisk Forenings politik er, således at der er mulighed for at følge den.
Man skal således være informeret om denne før man tiltræder sin post.
Der skal altid være mulighed for at låne et mødelokale på PF-gangen. På samme måde skal gangen være en god
arbejdsplads, der skal være brugbare PC'er og andre arbejdsredskaber, ligesom der skal være adgang til de nødvendige
informationer, såvel om dagen som om aftenen.
Der skal være et velfungerende sekretariat, der kan benyttes i forbindelse med administrative spørgsmål.
Der skal hvert år afholdes to opkvalificerende seminarer og politikskabende aktiviteter Indtil dags dato, styrende organ
seminar. Disse seminarer skal være henvendt til alle Polyteknisk Forenings medlemmer af styrende organer. Det er
bestyrelsens opgave at sørge for, at der bliver reklameret vidt og bredt for disse arrangementer.
