\subsection{Principper for fundraising i Polyteknisk Forening.}
I forbindelse med at øge Polyteknisk Forening’s indtjeningsmuligheder fremlægges hermed forslag til principper for fundraising i Polyteknisk Forening.\\

Det er essentielt at Polyteknisk Forening ikke lider økonomisk tab under fundraisingsprocessen. Derfor afholdes ingen udgifter, der måtte være forbundet med fundraiserens virke, og der er ikke mulighed for basisløn. Fundraiserens aflønnes mellem 5-15 \% provision af det fundraisede beløb. Lønudbetaling finder sted når pengene tilgår Polyteknisk Forening’s konto. Fundraiseren afholder selv alle udgifter i forbindelse med fundraisingen.\\

Det er væsentligt at fundraiseren, besidder et sådan format, at han kan repræsentere Polyteknisk Forening udadtil. Fundraiseren skal give relevant og korrekt information om Polyteknisk Forenings formål og om hvordan den modtagne donation vil blive brugt. Bestyrelsen skal sikre, at donationen så langt muligt bliver brugt i overensstemmelse med donors hensigt, indenfor en rimelig tid, og sikre yderligere en ansvarlig behandling af donationen, dets evt. tilhørende regnskab og resultatkontrollen med donationen.\\

Fundraiseren har ikke forhandlingsmandat, og kan således ikke indgå aftaler med donor, der måtte have indflydelse på Polyteknisk Forenings politikker, legitimitet eller andre forhold.\\

Fundraiseren skal holde en løbende dialog med bestyrelsen om hvilke virksomheder han er i kontakt med, og forholder sig til en ”Blacklist”, som bliver benyttet til andre formål af Polyteknisk Forening.\\

Fundraiseren forholder sig ligeledes til etisk retningslinier således Polyteknisk Forening med ro kan modtage donationer fra virksomheder.
\begin{itemize}
\item[-] Polyteknisk Forening søger ikke fundraising hos virksomheder der producerer våben. Her menes ikke
underleverandører ligesom der fokuseres på virksomhedens kerneaktiviteter.
\item[-] Polyteknisk Forening søger ikke fundraising hos virksomheder der bryder internationale konventioner om
menneske- og arbejdstagerrettigheder, miljøforhold samt korruption.
\end{itemize}

Der udfærdiges en ansættelseskontrakt til fundraiseren. Yderligere underskrives fortrolighedserklæring, og transportaftale (fundraiser underskriver at pengene altid indføres på Polyteknisk Forenings konto).