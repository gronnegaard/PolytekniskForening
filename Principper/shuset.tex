\section{Principper for S-huset}
S-huset skal gennem sin drift og arrangementer være til glæde og gavn for alle studerende ved DTU.
\subsubsection{Drift af S-huset}
\begin{itemize}
\item S-Husformanden og S-Husets bestyrer varetager i fællesskab den daglige drift af Studenterhuset, efter de af Fællesrådet afstukne retningslinier og inden for rammerne af PF's budget. Herunder prioritering af aktiviteter og fordeling af lokaler, samt at påse, at ordensreglementet overholdes og beslutte sanktioner overfor personer, der overtræder dette.
\item Kaffestuen skal holde åbent hver hverdag i 13-ugers perioden, eksamensperioden og i 3-uger perioden.
\item PF-Caféen skal som minimum holde åbent i 13-ugersperioden.
\item Kælderbaren skal så vidt muligt holde åbent 4 aftner om ugen i 13-ugers. Undtagelser kan ske i forbindelse med ferier, helligdage mv., hvor DTU holder lukket for undervisning.
\item Uregelmæssigheder i åbningstiderne annonceres mindst 14 dage i forvejen.
\item Der skal minimum afholdes rusjoint, vinterjoint plus 2 ekstra joints om året.\\
- en joint er defineret som et arrangement, hvor der er udskænkning andre steder end i Kælderbaren, efter at Kaffestuen har lukket.
\item Fredagsrock er gratis for alle.
\item PF-medlemmer skal i kraft af deres medlemskab have rabat på diverse entréer, ved køb af billetter i forsalg.
\item S-Husets bestyrer afgør suverænt alle forhold vedrørende udskænkning af drikkevarer.
\item S-Husformanden kan i samarbejde med bestyreren give karantæne fra S-Huset til personer der på den ene eller anden måde har opført sig forkert.
\item Ansatte på arbejde har pligt til at afvikle fester og arrangementer i S-Huset på forsvarlig vis.
\end{itemize}

\subsubsection{S-husets lokaler}
\begin{itemize}
\item Lokaler kan af S-Husledelsen stilles til rådighed for PF-klubber, subsidiært andre studentergrupperinger.
\item På dagen for joints og store arrangementer er der lukket for adgang til klublokalerne.
\item Der må ikke medbringes egne drikkevarer indenfor almindelige åbningstider i S-husets lokaler.
\item S-Husformanden og bestyreren kan give tilladelse til undtagelser.
\item S-Husets festlokaler kan stilles til rådighed for studerende på DTU udenfor almindelige åbningstider. Kælderbaren, Pejsestuen, Oticonsalen og Kaffestuen kan kun lejes med S-Husets bartendere. Læsesalen kan stilles til rådighed til arrangementer uden brug af S-Husets bartendere. Nærmere regler omkring lokalerne fastsættes af S-Husformanden i samarbejde med bestyreren.
\end{itemize}

\subsection{Principper for klubber under PF}
\begin{enumerate}
\item I Polyteknisk Forening er det besluttet kun at have klubber med socialt indhold eller hobbyklubber. Hvis der er tvivl vedr. indhold/formål tages dette op i klubledelsen.
\item Der skal som minimum være 7 medlemmer af klubben. Klubmedlemmer indskrives på PF-sekretariatet og klubmedlemskontingent opkræves af klubberne Minimum 2/3-dele skal være indskrevet studerende ved DTU.
\item Der kan i særlige tilfælde indstilles dispensation hvis en klub ikke formår at opretholde en medlemssammensætning på minimum 2/3-dele indskrevne studerende ved DTU.
\item Klubberne skal møde op til PF åbent hus mindst én gang om året.
\item Inden den 1. november skal der foreligge en skriftlig redegørelse fra Klubudvalgsformanden med hvilke klubber der har deltaget i PF åbent hus.
\item Klubbens årsberetning skal indeholde en redegørelse af aktivitetsniveau i klubben, samt en plan for hvervning af nye medlemmer i det kommende år. Disse dokumenter skal afleveres senest den 1. november til klubudvalgsformanden. Polyteknisk Forenings årsberetning skal indeholde klubbernes senest afleverede årsberetninger.
\item På det førstkommende møde, efter den 1. november behandler klubudvalget om klubber skal have frataget deres status som klub under Polyteknisk Forening. Endvidere kan klubudvalget diskutere om en klub skal fratages deres lokale faciliteter. Denne behandling sker på baggrund af redegørelse fra Klubudvalgsformanden jf. pkt. 5 og følgende fra hver enkelt klub:
\begin{itemize}
\item Regnskab
\item Årsberetning med redegørelse af aktivitetsniveau
\item Medlemsliste
\item Referat(er) fra generalforsamlinger
\end{itemize}
\item Klubledelsen afleverer efter Statusmødet(rne) en skriftlig redegørelse til Fællesrådet.
\end{enumerate}

\subsection{Vedtægter for klubber under Polyteknisk Forening}
\begin{list}
{\S \arabic{fpara}}{\usecounter{fpara}}
\item Klubbens navn er \makebox[2in]{\hrulefill}\\

\underline{Stk. 2}\\ 
Klubben er tilsluttet Polyteknisk Forening, og skal opfylde de standardregler, der er vedtaget af PF's
fællesråd, ifølge statutten \sref{S:okonomi:standardregler}
\item Klubbens formål er \makebox[2in]{\hrulefill}
\item Medlemsberettigede er personer over 18 år, som er eller har været studerende ved en højere læreanstalt, eller som er ansat ved Polyteknisk Forening eller DTU.\\

\underline{Stk. 2}\\
Personer der falder uden for disse kategorier, kan ansøge klubbens bestyrelse om medlemskab.\\

\underline{Stk. 3}\\
Der kan gives kontingents rabat til personer der tilhører følgende kategorier:
\begin{itemize}
\item Folk der er, eller har været medlem af PF.
\item Medlemmer af studenterorganisationer tilsluttet Danske Studerendes Fællesråd (DSF)
\item Ansatte i PF eller ved DTU.
\end{itemize}

\item Der skal som minimum være 7 medlemmer af klubben, hvoraf mindst 2/3-dele er indskrevne studerende ved DTU.

\item Generalforsamlingen er klubbens øverste myndighed. Alle personer med gyldigt medlemskab har tale- og stemmeret ved generalforsamlingen.\\

\underline{Stk. 2}\\
PF's procedureregler benyttes ved generalforsamlingen, hvis andet ikke vedtages.
\item Hvert år i tidsrummet fra den 1. februar til den 1. maj afholdes ordinær generalforsamling, der med angivelse af dagsorden skal indkaldes skriftligt, med 14 dages varsel.\\

\underline{Stk. 2} Den ordinære generalforsamlings dagsorden skal mindst indeholde følgende:
\begin{itemize}
\item Valg af dirigent.
\item Den afgående formands beretning.
\item Den afgående kasseres forelæggelse af regnskab.
\item Valg af formand
\item Valg af kasserer
\item Valg af bestyrelse.
\item Valg af revisorer.
\item Indkomne forslag.
\item Eventuelt.
\end{itemize}
Bestyrelsen konstituerer sig selv eller vælges direkte. Kassererens navn skal fremgå af referatet. Lovændringer skal være et særligt punkt på dagsordenen. Forslag under pkt. 6. skal være fremlagt i klublokalet senest 7 dage før den ordinære generalforsamling.\\

\underline{Stk. 3}\\
Den ordinære generalforsamling er beslutningsdygtig såfremt \makebox[0.4in]{\hrulefill}, dog mindst \makebox[0.4in]{\hrulefill} er tilstede. Er den ordinære generalforsamling ikke beslutningsdygtig, skal bestyrelsen inden 14 dage efter afholdelsen af ordinær generalforsamling indkalde til ekstraordinær generalforsamling med uforandret dagsorden. Denne generalforsamling vil være beslutningsdygtig uanset fremmøde.\\

\underline{Stk. 4}\\
Revisorerne må ikke være medlem af bestyrelsen.
\item Ekstraordinær generalforsamling skal med angivelse af dagsorden indkaldes skriftligt med mindst 14 dages varsel, når bestyrelsen finder det nødvendigt, eller når mindst 1/3 af medlemmerne ønsker det. Ekstraordinær generalforsamling må ikke afholdes udenfor DTU undervisningens 13 ugers periode.
\item Bestyrelsen består af formand, kasserer og mindst 3 andre medlemmer af klubben.
\item Formanden tegner klubben udadtil, og er ansvarlig i forholdet til Polyteknisk Forening. Formanden skal have været medlem af klubben i mindst 1 år.\\

\underline{Stk. 2}\\
Kassereren skal ligeledes have været medlem af klubben mindst 1 år. Kassereren tegner klubben økonomisk.\\

\underline{Stk. 3}\\
Der dannes naturlig præcedens for stk. 1 \& 2 ved opstart af en ny klub. Formand og kasserer skal her have været med fra starten.
\item Klubbens regnskaber føres af kassereren.\\

\underline{Stk. 2}\\ 
Regnskabs- og medlemsåret er fra 1. januar til den 31. december.\\

\underline{Stk. 3}\\ 
Klubbens medlemskartotek føres af en person fra bestyrelsen, hvis navn skal fremgå af referatet fra generalforsamlingen.\\

\underline{Stk. 4}\\ 
Klubbens nøglekartotek føres af en person fra bestyrelsen, hvis navn skal fremgå af referatet fra generalforsamlingen.\\

\underline{Stk. 5}\\
Kontingent fastsættes af bestyrelsen.\\

\underline{Stk. 6}\\
Nøgledepositum fastsættes af bestyrelsen.
\item Navneliste over bestyrelsen skal være opslået i klublokalet med angivelse af ansvarsområder.
\item Klubbens regnskab, årsberetning med redegørelse af aktivitetsniveau, medlemsliste og referat(er) fra generalforsamlinger skal hvert år fremsendes til S-Husformanden senest den 31.maj, som videregiver disse i Klubudvalget.
\item Referater af klubbens ordinære og eventuelt ekstraordinære generalforsamling(er), skal fremsendes til S-husledelsen.
\item Bestyrelsen kan midlertidigt suspendere et medlem af den pågældende klub, hvis bestyrelsen har belæg for at medlemmet har skadet klubben. Suspenderingen har højst virkning til førstkommende generalforsamling, hvor sagen tages op igen. En egentlig ekskludering kan kun ske på generalforsamlingen. Karantæne fra S-huset, udstedt af Klubudvalget, medfører at medlemmet suspenderes fra klubben i henhold til beslutningen taget af Klubudvalget.
\item Ændring af de klubspecifikke vedtægter kræver 2/3 af de ved generalforsamlingen fremmødte medlemmers stemmer.\\

\underline{Stk. 2} Vedtægtsændringer for klubben træder først i kraft efter godkendelse i Klubudvalget.\\

\underline{Stk. 3} Ordensregler kan klubbens bestyrelse fastsætte uden at indkalde til generalforsamling. Disse regler må ikke være i modstrid med vedtægterne.
\item Kun klubbens formand og kasserer er bemyndiget til at administrere klubbens økonomi i den daglige drift. Overdrages bemyndigelsen i tilfælde til andre er de stadig ansvarlige.
\item S-husets regler skal følges af alle klubmedlemmer.
\item Klubberne må ikke være basis for drift af forretning.
\item Klubmedlemmer har pligt til at kende vedtægter og reglement for klubben.
\item Ved klubbens ophør overdrages materialer og inventar til PF, såfremt det er betalt af Polyteknisk Forening. Klubbens formand er ansvarlig for det af PF overdragne materiel, samt for en eventuel tilbagelevering
\end{list}