\section{Behandling af økonomi i PF}
\label{kap:BehandlingOkonomi}
Da der historisk har været forskellige måder at behandle såvel budget samt andre finansielle punkter på, er disse
retningslinier nedskrevet.
Disse retningslinier er udtryk for diskussion i fællesrådet og tidligere lign. oplæg.
\\
\\
\textbf{Definitioner}
Gangen – Herunder menes budget for studentersociale- og studenterpolitiskeaktiviteter samt selve gangen. Dog ligges
studiestart også herunder. Disse budgetter kaldes bevillingsbudgetter, da de ikke er overskudsgivende el. har finansielle
indtægter.
S-Huset – Herunder ligger dels Skallen, Kaffestuen, Kælderbaren, PF-Cafeen og endelig Scenelys. Alle disse afdelinger
er finansbudgetter, hvilket betyder de har salg, løn, varekøb og andre variable poster.
PFS, FællesIT, PIT – Disse er alle administrative afdelinger som ligesom S-Huset er finansbudgetter.
Fortrolige oplysninger – Herunder forstås alle personlige oplysninger, fx løn til ansatte. Endvidere er det også vigtigt at
omsætningstal for dels PIT, Scenelys og S-Huset generelt af konkurrencehensyn holdes fortroligt. Jf. PF’s love og
statutter kan møder lukkes, ved enkelt drøftelse af særlig fortrolig karakter.
Behandling af budgettet for PF
I forbindelse med behandling af budgettet for PF, skal det deles op i 3 dele;
\begin{itemize}
\item Før budgetmødet
\begin{enumerate}
\item 
Faglige råd
De faglige råd har mulighed for at behandle budgettet for gangen og prioritere de enkelte poster.
Derfor skal de faglige råd have budgettet senest en uge før budgetmødet, så de kan diskutere budgettet
i de enkelte faglige råd. Som tillæg til dette budget, vil en oversigt over de enkeltes afdelingers
budgetresultat blive vedlagt. Denne oversigt vedlægges for at give de faglige råd et billede af hvordan
budgettet ser ud, hvilket giver forståelse for den større sammenhæng i forbindelse med
budgetbehandlingen i det enkelte faglige råd.
Det er ligeledes bestyrelsens ansvar at deltage i alle faglige råds møder og kunne redegøre for de
grundlæggende linier og prioriteringer, mens detaljerede spørgsmål skal kunne besvares af
Forretningsrådsformanden.
b.
\item Fællesrådsrepræsentanterne skal ligeledes kunne hente det samlede budget, senest en uge før
budgetmødet. På denne måde har repræsentanterne bedre mulighed for at forberede sig til mødet, der
heller ikke skal trækkes ud med lange læsepauser.
\end{enumerate} 
\item Behandling af selve budgettet
\begin{enumerate}
\item Budgetmødet startes med en gennemgang af det samlede budget, bortset fra selve gangen. Her kan
fællesrådet spørge ind til enkelte dele af budgettet og diskutere generelle dispositioner, pga. praktiske
årsager, er det yderst uhensigtsmæssigt hvis fællesrådet ændrer på de enkelte dele.
På denne måde har fællesrådet fået en fornemmelse af budgettet og de enkelte dele før fastlæggelse af
bundlinien.
\item Derefter tages en indledende diskussion omkring bundlinien. Bundlinien skal her forstås som
bundlinien for gangen. Naturligvis vil en bundlinie for gangen direkte påvirke den endelig bundlinie
for budgettet.
\item Budgettet for gangen gennemgås. Hver enkelt post gennemgås enkeltvis ved afstemning. Det er her
muligt for de enkelte råd at komme med ændringer til de enkelte poster. Dette klares praktisk ved at
splitte hver post op i forskellige puljer, hvilket gør at der således skal stemmes om hver enkelt pulje til
posten. Der er tre forskellige slags stemmer, alt efter prioritering

\begin{list}{•}
\item Høj: (værdi 1) Denne stemme gives til de poster, man mener skal prioriteres højt i budgettet
\item Blank: (værdi 0) Denne stemme gives, hvis man mener posten er passende og har middel vægt i budgettet.
\item Lav: (værdi -1) Denne stemme gives, hvis man mener posten er for høj eller slet ikke hører hjemme i budgettet.
\end{list}
Dog skal det nævnes, at nogle poster ikke kan ændres grundet kontrakt- eller aftalemæssige
forpligtelser.
\item Efter gennemgang og afstemning til hver enkelt post bliver posterne opstillet i prioriteret rækkefølge.
Ved at addere hver enkel post vil man på denne måde kunne slå en streg der hvor bundlinien rammes.
Herefter skal fællesrådet dog kigge en ekstra gang på prioriteringerne da det kan ske, at nogle ting er
røget under linien uhensigtsmæssigt og ligeledes den anden vej. Det kan ved denne gennemgang
besluttes at ændre bundlinien.
Endelig bliver fællesrådet enige om budgettet for gangen.
\end{enumerate}
\item Efter budgetmødet\\ 
Når det endelig budget er godkendt i fællesrådet, har fællesrådsrepræsentanterne naturligvis mulighed til at gå tilbage til deres respektive faglige råd med et detaljeret budget over gangen, men også oversigten over de enkelte dele. På denne måde kan de enkelte faglige råd få overblik over den endelig budgetterede bundlinie for. Det endelige detaljerede budget vil dog fortsat være fortroligt.
\end{itemize}
\textbf{Godkendelse af Årsrapport}\\
\\
Årsrapporten (tidl. årsregnskab) gennemgås af en uvildig statsautoriseret revisor og skal ifølge vores egne vedtægter
godkendes af fællesrådet inden 1. december. Årsrapporten skal være tilgængelig for fællesrådet senest 1 uge før mødet,
og behandles som lukket.
Det er ikke muligt for fællesrådet at ændre i årsrapporten.
Under godkendelse af årsrapporten vil de enkelte faglig råds repræsentanter kunne komme med uddybende spørgsmål
til årsrapporten, som vil kunne besvares af Forretningsrådsformanden.
Herefter skrider fællesrådet til godkendelse af årsrapporten.\\
\\
\textbf{Godkendelse af internt regnskab}\\

Det interne regnskab er et billede af hvordan indtægter og udgifter er blevet fordelt i regnskabsårets løb. Det interne
regnskab vil spejle sig i dels det vedtagne budget, men også den reelle bogføring der er foretaget.
For det interne regnskab gælder det, som for budgettet, at kun gangen kan godkendes i de faglige råd, mens en oversigt
over de enkelte afdelinger vedlægges dette begrænsede regnskab.
Det er ikke muligt for de faglige råd at ændre i regnskabet, da det er jo er et billede af hvordan det er gået.
Dog kan det interne regnskab med fordel bruges til at kigge på hvor godt budgettet er blevet overholdt, samt give en
indikation af hvorledes midlerne fordeler sig. Derfor vil en gennemgang af det interne regnskab være en god øvelse for
fællesrådet såvel som den kommende forretningsrådsformand.
Selve gennemgangen af det interne regnskab tages afdeling for afdeling, og er fortroligt for alle afdelinger bortset fra
gangen.
Efter endt gennemgang skrider fællesrådet til godkendelse af det interne regnskab.
Behandling af kritisk revision
Reglerne for godkendelse af den kritiske revision er de samme som for godkendelse af årsrapporten, altså 1. december,
og den kritiske revision skal kunne udleveres senest 1 uge før mødet.
Der har tidligere i fællesrådet været ønske om at kunne fremvise dele af den kritiske revision for de faglige råd. Dette
være sig generelle ting om PF’s økonomi, beskrivelse af gennemgang og generelle stillinger ang. foreningen. Derfor vil
den kritiske revision bestå af en åben og en lukket del.
Den kritiske revision bliver gennemgået afsnit for afsnit, med de kritiske revisorer tilstede, i den udstrækning det kan
lade sig gøre. Her vil de kunne uddybe den kritiske revision såfremt noget er uklart, eller repræsentanterne fra de faglige
råd har spørgsmål.
51
Her vil bestyrelsen, og i særdeleshed Forretningsrådsformanden, blive brugt til at svare på konkrete problemstillinger og
henstillinger i den kritiske revision.
Ligesom regnskabet er den kritiske revision en status over det tidligere regnskabsår, hvorfor indholdet ikke direkte kan
ændres. Dog kan fællesrådet få fjernet dele de ikke finder sandt eller hensigtsmæssigt at have stående i den kritiske
revision.
Der opfordres på det kraftigste til at den kritiske revision bliver brugt i forbindelse med budgetlægningen, således gode
råd og anbefalinger, problemstillinger og andre overvejelser kan tages med i budgetlægningen. På denne måde sikres
overleveringen bedre år for år, end den har gjort tidligere.
Efter endt gennemgang af den kritiske revision godkendes denne af fællesrådet.
Behandling af andre økonomiske punkter
Ved behandling af andre økonomiske punkter er det meget afhængig af det enkelte punkt, hvordan det skal behandles.
Typer af punkter som bør behandles fortroligt:
\begin{list}{•}
\item Personfølsomme oplysninger indeholdt
\item Konkurrencemæssige oplysninger
\item Forhandlingsmæssige punkter
\item Orienteringspunkter om andet end gangens økonomiske tilstand
\end{list}
Typer af punkter som ikke bør behandles fortroligt:
\begin{list}{•}
\item Budgetoverskridelser på gangen
\item Budgetændringer på gangen
\item Orienteringspunkter
\end{list}
Da de faglige råd vil have mulighed for at behandle såvel budget, regnskab, kritisk revision for gangen og tilhørende
information, skal denne slags information ikke behandles fortroligt da det berører alle medlemmer af PF.
Resten skal dog behandles fortroligt af hensyn til ansatte, konkurrencemæssige forhold og forhandlingsmæssige
mandater/forhold.
\\
\\

\textit{Vedtaget på FR152}
