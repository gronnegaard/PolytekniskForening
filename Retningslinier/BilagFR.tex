\section{Retningslinjer for bilag til fællesrådet.}

Bilagene til fællesrådsmøderne har været af svingene kvalitet, derfor skal nedenstående regler følges for at sikre kvaliteten.

\begin{itemize}
\item Som udgangspunkt bør alle punkter på fællesrådsmøderne være ledsaget af et bilag. Grunden til dette er, at fællesrådets arbejde af demokratiske grunde skal være så gennemsigtigt for resten af foreningen som muligt.
Der findes dog situationer hvor det ikke vil være nødvendigt med et bilag, kontakt FRFU for at afklare om det er nødvendigt i den aktuelle sag.
\item Et bilag skal altid forsynes med: Dato, forfatter(e), sidetal og et bilagsnummer som tildeles af FRFU.
\item Alle forkortelser bør være skrevet fuldt ud mindst en gang. Ved brug af mindre almindelig forkortelser kan en forklaring være en god ide.
\item Når et bilag, der forventes behandlet i de faglige råd laves, skal der tænkes på, at de personer der skal behandle det typisk har mindre erfaring end fællesrådsmedlemmerne. Derfor skal problemstillingen og indviklede koncepter forklares ekstra grundigt.
\item Diskussionsoplæg og ændringsforslag bør indeholde en introducerende tekst, der forklarer baggrunden og motivationen for diskussionen eller ændringen.
\item Bilag der omhandler ændringer i eksisterende tekst, skal udformes således at det klart fremgår hvad der er ændret. Dette kan f.eks. gøres ved at tilføje det gamle tekststykke ovenover det nye, gerne med forskellig typografi.
\item Ved udformningen af diskussionsoplæg kan det være en god idé at høre alle interessenter, således at oplægget
kan indeholde en diskussion af fordele og ulemper ved det forslåede.
\item Ved udformningen af bilag, kan FRFU kontaktes for en vurdering af om bilaget er af tilfredsstillende kvalitet.
\end{itemize}