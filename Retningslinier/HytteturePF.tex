\section{Retningslinjer for PF hytteture}
I det følgende er der skelnet mellem tre slags hytteture;
\begin{itemize}\addtolength{\itemsep}{-0.5\baselineskip}
\item Studiestartsture
\item Hytteture med uddannelse som formål
\item Hytteture med et socialt formål
\end{itemize}

De generelle retningslinjer gælder naturligvis for alle tre former for hytteture (og andre hytteture i PF-regi, hvis de
skulle opstå).

\subsubsection{Generelle retningslinjer}
\begin{itemize}\addtolength{\itemsep}{-0.5\baselineskip}
\item Der skal på enhver hyttetur i PF-regi forefindes mindst en komplet førstehjælpskasse.
\item Der skal som det første ved ankomst til hytten undersøges hvilket brandslukningsmateriel samt hvilke
flugtveje, der findes og hvor.
\item Der skal blandt arrangørerne af turen være en ansvarlig for hvert af de ovenstående to punkter.
\item Der skal på turen være minimum en dagsansvarlig. De(n) dagsansvarlige har til et hvert tidspunkt det overordnede
ansvar for alle igangværende aktiviteter.
\item Dagsansvarlige er til en hver tid ædru og i stand til at tage ansvar, det betyder som minimum at man ikke har indtaget nogen former for alkohol i de foregående 16 timer.
\item Der skal før afrejse gøres rent efter anvisningerne i hytten eller lejekontrakten.
\item Der kan i forbindelse med billeje være forskellige regler for f.eks. chaufførens alder. Det er i den forbindelse chaufførens ansvar at disse regler overholdes.
\end{itemize}

\subsubsection{Retningslinjer specielt for studiestartsture}
\begin{itemize}\addtolength{\itemsep}{-0.5\baselineskip}
\item En studiestartstur skal have godkendt et visionsoplæg af deres overordnede, koordinerende gruppe. Dette
visionsoplæg skal blandt andet indeholde arrangørernes tanker om hvilke grænser, der er på turen (f.eks. mht.
nøgenhed, alkohol, fysisk fare).
\item Der skal minimum være en dagsansvarlig blandt arrangørerne. Derudover skal der minimum være en kørselsansvarlig blandt arrangørerne. Kørselsansvarlig er underlagt samme generelle retningslinjer som dagsansvarlig og skal derudover være i stand til at køre bil.
\end{itemize}

\subsubsection{Retningslinjer specielt for uddannelsesture}
\begin{itemize}\addtolength{\itemsep}{-0.5\baselineskip}
\item Der er på forhånd kommunikeret klart ud til deltagerne, hvad der forventes af dem på turen.
\item Arrangørerne udarbejder på forhånd en plan for turen til internt brug. Planen skal gøre rede for det faglige
indhold samt indeholde et detaljeret dagsprogram med beskrivelse af aktiviteter, oversigter over workshops
indeholdende deltagerlister, tidsplaner og geografi.
\end{itemize}
