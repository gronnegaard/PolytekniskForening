\section{Retningslinjer for referater for PF’s Fællesråd}
\begin{list}{•}
\item Fællesrådets Forretningsudvalg søger for, at der udarbejdes et referat af møderne i Fællesrådet, som godkendes
        på det efterfølgende ordinære møde. Der skal ved mødets afslutning afleveres en kopi af det foreløbige referat
             til Fællesrådets Forretningsudvalg.
\item Referatet skal som minimum indeholde:
\begin{list}{-}
\item oplysninger om tid og sted for mødets afholdelse
\item dagsorden for mødet
\item tilstedeværende personer under mødet
\item hvad der blev besluttet under de enkelte punkter
\item resultatet af eventuelle afstemninger med angivelse af, hvor mange stemmer der blev afgivet for eller imod forslaget, og hvor mange der undlod at stemme
\item en angivelse af bilag til mødet
\end{list}
\item Referatet kan udformes som citatreferat, hvor enkelte udtalelser citeres direkte, eller diskussionsreferat, hvor enkelte udtalelser ikke refereres men de gennemgående synspunkter under debatten refereres. Ethvert medlem kan forlange sin afvigende  mening kort optaget i referatet.

\end{list}

\textit{Vedtaget på FR152}
